%\documentstyle[11pt,epsfig,psfig,times]{article} 

%\textheight 9in
%\topmargin 1in
%\footheight 0.5in
%\textwidth 6.8in
%\oddsidemargin -0.2in

%\usepackage{latex8}
%\usepackage{psfig}
%\usepackage{epsfig}
%\usepackage{latexsym}
%\usepackage{amssymb}

%\begin{document}

Let ${\tt preset}(t)$ and ${\tt postset}(t)$ denote the preset and postset of
a transition $t$.  ${\tt preset}(p)$ and ${\tt postset}(p)$ are similarly 
defined for a place $p$.   


\subsection{Transform 0: Merge Parallel Places}
\label{reduce0}

\begin{figure}[tbh]
\begin{center}
\begin{tabular}{cc}
\psfig{figure=xform0-a,width=27.5mm} \hspace{5mm} &
\psfig{figure=xform0-b,width=22.5mm} \\
(a) \hspace{6mm} & (b)
\end{tabular}
{\caption{\label{xform0}Transform 0. Merge parallel places.}}
\end{center}
\end{figure}

Figure~\ref{xform0}(a) shows two places in parallel.  They can be
merged into a single place as shown in Figure~\ref{xform0}(b) if the following requirements hold:
\begin{itemize}
\item {\tt preset(p1) = preset(p2)}.  
\item {\tt postset(p1) = postset(p2)}.  
\item {\tt marking(p1) = marking(p2)}
\end{itemize}

%\clearpage

\subsection{Transform 1: Remove a Place in a Self Loop}
\label{reduce}

\begin{figure}[!tbh]
\begin{center}
\begin{tabular}{cc}
\psfig{figure=xform1-a,width=31.25mm} \hspace{5mm} &
\psfig{figure=xform1-b,width=35mm} \\
(a) \hspace{6mm} & (b)
\end{tabular}
{\caption{\label{xform1}Transform 1. Remove a self loop on a
    transition.}}
\end{center}
\end{figure}

Figure~\ref{xform1}(a) shows a transition $t$ with a self loop, where a 
place $p$ is in ${\tt preset}(t)$ and ${\tt postset}(t)$, and is marked.
The transformation removes $p$ and its incoming and outgoing flow relations.
Then, the upper bound delay of the flow relations entering $t$ are 
modified such that the upper bound is the maximum of the current max
and that of the flow relation between $p$ and $t$.
The new TPN is shown in Figure~\ref{xform1}(b).  If there is only the
same one place in the preset and postset of $t$, this place is not
removed as shown in Figure~\ref{bad}(c). 

%\clearpage 

\subsection{Transform 2: (BAD TRANSFORM) Remove a Transition in a Loop}
\label{reduce1}

\begin{figure}[tbh]
\begin{center}
\begin{tabular}{cc}
\psfig{figure=xform2-a,width=27.5mm} \hspace{5mm} &
\psfig{figure=xform2-b,width=22.5mm} \\
(a) \hspace{6mm} & (b)
\end{tabular}
{\caption{\label{xform2}Transform 2. Remove a transition in a loop.}}
\end{center}
\end{figure}

Figure~\ref{xform2}(a) shows a loop formed by a transition $t_3$, where the size
of the preset and postset of $t_3$ is 1. The semantics
of this TPN can be interpreted as transition $t_4$ can fire after the
infinite number of times of firings of transition $t_2$ and $t_3$.
If ${\tt u}(p_1) + {\tt u}(p_2) > 0$, by changing ${\tt u}(p2)$ to $\infty$, 
the timing of $t_4$ is preserved.  The new TPN after removing transition 
$t_3$ and its incoming and outgoing flow relations is shown in 
Figure~\ref{xform2}(b).

The code implemented the above transformation is in the function call 
{\em xform\_1} in postproc.cc.  

%\clearpage

\subsection{Transform 3: Remove a Transition With a Single Place in the Postset}
\label{reduce2}

\begin{figure}[tbh]
\begin{center}
\begin{tabular}{cc}
\psfig{figure=xform3a-a,width=35mm} \hspace{10mm} &
\psfig{figure=xform3a-b,width=50mm} \\
(a) \hspace{10mm} & (b)
\end{tabular}
{\caption{\label{xform3a}Transform 3a. Remove the transition with
    the size of postset of 1.}}
\end{center}
\end{figure}

\begin{figure}[tbh]
\begin{center}
\begin{tabular}{cc}
\psfig{figure=xform3b-a,width=37.5mm} \hspace{10mm} &
\psfig{figure=xform3b-b,width=62.5mm} \\
(a) \hspace{10mm} & (b)
\end{tabular}
{\caption{\label{xform3b}Transform 3b. The transformation in
    Figure~\ref{xform3a} without the restriction that places in preset(t)
    have postset sizes of 1.}}
\end{center}
\end{figure}

\begin{figure}[tbh]
\begin{center}
\begin{tabular}{cc}
\psfig{figure=xform3c-a,width=35mm} \hspace{10mm} &
\psfig{figure=xform3c-b,width=57.5mm} \\
(a) \hspace{10mm} & (b)
\end{tabular}
{\caption{\label{xform3c}Transform 3c.  The transformation in
    Figure~\ref{xform3a} without the restriction that the place in
    postset(t) have preset size of 1.}}
\end{center}
\end{figure}

Figure~\ref{xform3a}(a) shows a transition $t$ with only one place in its
postset.  Suppose $t$ is removed.  The incoming and outgoing flow relations
of $t$ are also removed.  Then the place in ${\tt postset}(t)$ is 
merged with those in ${\tt preset}(t)$.  The delay bounds of the place in 
${\tt postset}(t)$ are added to the delay bounds of the places in 
${\tt preset}(t)$, as shown in Figure~\ref{xform3a}(b).
The transformation in Figure~\ref{xform3a}  has the restrictions that
for each place $p$ in
${\tt preset}(t)$, the size of ${\tt postset}(p)$ equals 1 and  for the
place $p$ in ${\tt postset}(t)$, the size of ${\tt preset}(p)$ equals
1.  Figures \ref{xform3b} and \ref{xform3c} show the same
transformations without each restriction respectively.

Figure~\ref{bad}(a) and (b) show where these above transformations
cannot be applied.

All places in preset(t) must have same marking (all marked or all
unmarked) and place in postset(t) can be marked.  If either preset(t)
or postset(t) is marked, new combined places are marked.

%\clearpage

\subsection{Transform 4: Remove A Transition With A Single Place in the Preset}
\label{xform-3}

\begin{figure}[tbh]
\begin{center}
\begin{tabular}{cc}
\psfig{figure=xform4a-a,width=40mm} \hspace{10mm} &
\psfig{figure=xform4a-b,width=41.25mm} \\
(a) \hspace{10mm} & (b)
\end{tabular}
{\caption{\label{xform4a}Transform 4a. Remove a transition with
    preset of size 1.}}
\end{center}
\end{figure}

\begin{figure}[tbh]
\begin{center}
\begin{tabular}{cc}
\psfig{figure=xform4b-a,width=37.5mm} \hspace{10mm} &
\psfig{figure=xform4b-b,width=40mm} \\
(a) \hspace{10mm} & (b)
\end{tabular}
{\caption{\label{xform4b}Transform 4b. Transformation in Figure~\ref{xform4a}
    without restriction that the places in preset(t) have a preset of
    size 0.}}
\end{center}
\end{figure}

\begin{figure}[tbh]
\begin{center}
\begin{tabular}{cc}
\psfig{figure=xform4c-a,width=40mm} \hspace{10mm} &
\psfig{figure=xform4c-b,width=45mm} \\
(a) \hspace{10mm} & (b)
\end{tabular}
{\caption{\label{xform4c}Transform 4c. Transformation in Figure~\ref{xform4a}
    without restriction that the places in postset(t) each have a preset of
    size 1.}}
\end{center}
\end{figure}

Figure~\ref{xform4a}(a) shows a transition $t$ with only one place in its
preset.  Suppose $t$ is removed.  The incoming and outgoing flow relations
of $t$ are also removed.  The the place in ${\tt preset}(t)$ is 
merged with those in ${\tt postset}(t)$.  The delay bounds of the place in 
${\tt preset}(t)$ are added to the delay bounds of the places in 
${\tt postset}(t)$, as shown in Figure~\ref{xform4a}(b).
The transformation in Figure~\ref{xform4a} as the restrictions that
for the place $p$ in
${\tt preset}(t)$, the size of ${\tt postset}(p)$ equals 1 and for each
place $p$ in ${\tt postset}(t)$, the size of ${\tt preset}(p)$ equals
1, too.  Figures~\ref{xform4b} and \ref{xform4c} demonstrate the same
transform where these restrictions are loosened.

The above transformations are also 
restricted by the conditions shown in Figure~\ref{bad}(a) and (b).

All places in postset(t) must have same marking (all marked or all
unmarked) and place in preset(t) can be marked.  If anything is
marked, new combined places are marked.

%\clearpage

\subsection{Transform 5: Merge Transitions With the same preset and postset}
\label{merge-1}

\begin{figure}[tbh]
\begin{center}
\begin{tabular}{cc}
\psfig{figure=xform5a-a,width=58.75mm} \hspace{10mm} &
\psfig{figure=xform5a-b,width=48.75mm} \\
(a) \hspace{10mm} & (b)
\end{tabular}
{\caption{\label{xform5a}Transform 5a. Merge transitions with the same
    preset and postset}}
\end{center}
\end{figure}

\begin{figure}[tbh]
\begin{center}
\begin{tabular}{cc}
\psfig{figure=xform5b-a,width=48.75mm} \hspace{10mm} &
\psfig{figure=xform5b-b,width=46.25mm} \\
(a) \hspace{10mm} & (b)
\end{tabular}
{\caption{\label{xform5b}Transform 5b. Merge transitions where the
    places in their preset and postset have the same preset and
    postset.}}
\end{center}
\end{figure}

Figure~\ref{xform5a}(a) shows two transitions $t_6$ and $t_7$ with 
the same preset and postset.  Those two transitions can be merged together
without affecting the behavior on transitions $t_{10}$, $t_{11}$, and 
$t_{12}$ as shown in Figure~\ref{xform5a}(b).  Since no places are
being removed, no marking restrictions apply.

%Note: Transform 5b is not turned on by default.  It was found to have
%minimal effect except for in a few examples and was computationaly
%intensive.  Additionally, the correctness of this transformation in
%its current form is in question.

Another case is where the two transitions do not have the same preset
and postset, but the places in their preset and postset have the same
preset and postset as shown in Figure~\ref{xform5b}(a).  The two
transitions can still be merged along with the places in their preset
and postset. This transformation is depicted in Figure~\ref{xform5b}.
In order for the transformation to be applied certain timing
constraints must be satisfied.  Specifically, arcs between the places
in the postset to transitions in the postsets of the postset must have
the same timing bounds.  Furthermore, the place to transition arcs
feeding the transitions to be merged must have the same timing
constraints.

%There is also requirement on this transformation that the places in
%the preset of the merged two transitions must have same postset
%excluding the merging transitions, and the places in the postset of
%the merged two transitions must have same preset excluding the merging
%transitions.  For example, in Figure~\ref{xform5b}(a), the places in
%the preset of $t_3$ and $t_4$ are $p_1$ and $p_2$.  The postset of
%$p_1$ and $p_2$ contains only $t_2$, excluding $t_3$ and $t_4$.  The
%places in the postset of $t_3$ and $t_4$ are $p_3$ and $p_4$.  The
%preset of $p_3$ and $p_4$ contains only $t_5$, excluding $t_3$ and
%$t_4$.

%\clearpage

\subsection{Transform 6: Merge Transitions With the same preset}
\label{merge-2}

\begin{figure}[tbh]
\begin{center}
\begin{tabular}{cc}
\psfig{figure=xform6-a,width=57.5mm} \hspace{5mm} &
\psfig{figure=xform6-b,width=50mm} \\
(a) \hspace{5mm} & (b)
\end{tabular}
{\caption{\label{xform6}Transform 6. Merge transitions with the same
    preset.}}
\end{center}
\end{figure}

Figure~\ref{xform6}(a) shows two transitions $t_1$ and $t_2$ with 
the same preset.  It is also required that the places in the postset of 
$t_1$ and $t_2$ must have the same preset aside from the two
transitions being merged.  Additionally, there can only be one place
in the postsets of t1 and t2.  Those two 
transitions can be merged together without affecting the behavior on 
transitions $t_3$, $t_4$, and $t_5$ as shown in Figure~\ref{xform6}(b).

Since the places in the postset of the merging transitions will be
merged, they must have a consistent marking.  The newly merged place
will have the same marking as the places being merged.
%\clearpage

\subsection{Transform 7: Merge Transitions With the same postset}
\label{merge-3}

\begin{figure}[tbh]
\begin{center}
\begin{tabular}{cc}
\psfig{figure=xform7-a,width=51.25mm} \hspace{10mm} &
\psfig{figure=xform7-b,width=51.25mm} \\
(a) \hspace{10mm} & (b)
\end{tabular}
{\caption{\label{xform7}Transform 7. Merge transitions with the same
    postset.}}
\end{center}
\end{figure}

Figure~\ref{xform7}(a) shows two transitions $t_1$ and $t_2$ with 
the same postset.  It is also required that the places in the preset of 
$t_1$ and $t_2$ must have the same postset aside from $t_1$ and $t_2$.
Those two 
transitions can be merged together without affecting the behavior on 
transitions $t_3$, $t_4$, and $t_5$ as shown in
Figure~\ref{xform7}(b).  Additionally, the transitions being merged
may each only have one place in their presets.  The places in the
preset of the transitions
being merged should have consistent markings.  The newly merged place
in the transformed net should have the same marking as the places that
are merged.



%\clearpage

\subsection{Illegal Transformations}

\begin{figure}[tbh]
\begin{center}
\begin{tabular}{cc}
\psfig{figure=bad-1,width=25mm} \hspace{5mm} &
\psfig{figure=bad-2,width=32.5mm} \\
(a) \hspace{6mm} & (b) \\
\psfig{figure=bad-3,width=20mm} \hspace{5mm} &
\psfig{figure=bad-4,width=12.5mm} \\
(c) \hspace{6mm} & (d)
\end{tabular}
{\caption{\label{bad}Illegal transformations.}}
\end{center}
\end{figure}

Figure~\ref{bad} shows several illegal transformations. In Figure~\ref{bad}(a),
neither $t_1$ nor $t_2$ can be removed using transformations in 
section~\ref{reduce1} and \ref{reduce2}.  
They may be merged by transformations insection~\ref{merge-1}, 
\ref{merge-2}, or \ref{merge-3}.  In Figure~\ref{bad}(b), $t_3$ cannot be
removed in that it is impossible to have two places connected without a
transition in between. Figure~\ref{bad}(c) shows a stand-alone self loop which
is not removed for some reason.  I think it can be removed if the wire of 
that transition is not used in any levels.  Figure~\ref{bad}(d) shows a
dangling transition $t_2$.  It cannot be removed because safety failure
maybe hidden; otherwise.

%\clearpage

\subsection{To Do and Notes}

\begin{itemize}
\item Transforms 3d and 4d.  Same As 3a and 4a but with both
  restrictions removed.
\end{itemize}

%\end{document}