%%%%%%%%%%%%%%%%%%%%%%%%%%%%%%%%%%%%%%%%%%%%%%%%%%%%%%%%%%%%%%%%%%%%%%%%%%%%%%%
%% Copyright (c) 2009 Kevin Jones <kevinj@eng.utah.edu>
%%
%% File Name     : abstraction.tex
%% Author        : Kevin Jones
%% E-mail        : kevinj@eng.utah.edu
%% Date Created  : 05/18/2009
%%
%% Description   : 
%%
%% ToDo          : Add some sort of index.
%%
%%%%%%%%%%%%%%%%%%%%%%%%%%%%%%%%%%%%%%%%%%%%%%%%%%%%%%%%%%%%%%%%%%%%%%%%%%%%%%%

\documentclass[titlepage,11pt]{article}
\textwidth 6.5in
\textheight 9in
\oddsidemargin -0.2in
\topmargin -0.5in
\usepackage{indentfirst,graphics,alltt,epsfig,color}

\title{Labeled Hybrid Petri Net Simplification Transforms}
\date{Created: May 18, 2009\\
  Last Revised: June 17, 2009
}
%\author{Kevin Jones}

\begin{document}
  \maketitle

  %show only subsection granularity in the toc
  \setcounter{tocdepth}{2} 

  \tableofcontents \newpage
  
   \setlength{\parindent}{0em} \setlength{\parskip}{10pt}
  
 Let ${\tt preset}(t)$ and ${\tt postset}(t)$ denote the preset and postset of
a transition $t$.  ${\tt preset}(p)$ and ${\tt postset}(p)$ are similarly 
defined for a place $p$.   

 This document describes the merging of assignments to enable transitions to be
removed without interfering with the validity of the verification.  These
transforms create a superset of the original behavior.  Verification based on
the abstracted nets may include false negatives but will not include false 
positives.  For each of these transforms, the removal or movement of assignments
allows other transforms to be applied that will remove unnecessary transitions.

\section{Graph Manipulation}

 This section details the basic graph transformations that can be made to a
labeled hybrid petri net (LHPN).  These transformations were designed to be 
applied to single transitions graph (STG) nets.  They can only be legally
applied if the transition to be removed does not have any assignment or
enabling condition.  Any change in the delays is shown in the figures.

\subsection{Transform 0: Merge Parallel Places}
\label{reduce0}

\begin{figure}[tbh]
\begin{center}
\begin{tabular}{cc}
\scalebox{0.5}{\input{figs/fig0-a.figtex}}
\scalebox{0.5}{\input{figs/fig0-b.figtex}} \\
(a) \hspace{6mm} & (b)
\end{tabular}
{\caption{\label{xform0}Transform 0. Merge parallel places.}}
\end{center}
\end{figure}

Figure~\ref{xform0}(a) shows two places in parallel.  They can be
merged into a single place as shown in Figure~\ref{xform0}(b) if the following requirements hold:
\begin{itemize}
\item {\tt preset(p1) = preset(p2)}.  
\item {\tt postset(p1) = postset(p2)}.  
\item {\tt marking(p1) = marking(p2)}
\end{itemize}

%\clearpage

\subsection{Transform 1: Remove a Place in a Self Loop}
\label{reduce}

\begin{figure}[!tbh]
\begin{center}
\begin{tabular}{cc}
\scalebox{0.5}{\input{figs/fig1-a.figtex}}
\scalebox{0.5}{\input{figs/fig1-b.figtex}} \\
(a) \hspace{6mm} & (b)
\end{tabular}
{\caption{\label{xform1}Transform 1. Remove a self loop on a
    transition.}}
\end{center}
\end{figure}

Figure~\ref{xform1}(a) shows a transition $t$ with a self loop, where a 
place $p$ is in ${\tt preset}(t)$ and ${\tt postset}(t)$, and is marked.
The transformation removes $p$ and its incoming and outgoing flow relations.
The new LHPN is shown in Figure~\ref{xform1}(b).  If there is only the
same one place in the preset and postset of $t$, this place is not
removed as shown in Figure~\ref{bad}(c). 

%\clearpage 

\subsection{Transform 2: (BAD TRANSFORM) Remove a Transition in a Loop}
\label{reduce1}

\begin{figure}[tbh]
\begin{center}
\begin{tabular}{cc}
\scalebox{0.5}{\input{figs/fig2-a.figtex}}
\scalebox{0.5}{\input{figs/fig2-b.figtex}} \\
(a) \hspace{6mm} & (b)
\end{tabular}
{\caption{\label{xform2}Transform 2. Remove a transition in a loop.}}
\end{center}
\end{figure}

Figure~\ref{xform2}(a) shows a loop formed by a transition $t_3$, where the size
of the preset and postset of $t_3$ is 1. The semantics
of this LHPN can be interpreted as transition $t_4$ can fire after the
infinite number of times of firings of transition $t_2$ and $t_3$.
If ${\tt u}(p_1) + {\tt u}(p_2) > 0$, by changing ${\tt u}(p2)$ to $\infty$, 
the timing of $t_4$ is preserved.  The new LHPN after removing transition 
$t_3$ and its incoming and outgoing flow relations is shown in 
Figure~\ref{xform2}(b).

The code implemented the above transformation is in the function call 
{\em xform\_1} in postproc.cc.  

%\clearpage

\subsection{Transform 3: Remove a Transition With a Single Place in the Postset}
\label{reduce2}

\begin{figure}[tbh]
\begin{center}
\begin{tabular}{ccc}
\scalebox{0.5}{\input{figs/fig3a-a.figtex}}
\scalebox{0.5}{\input{figs/fig3a-b.figtex}}
\scalebox{0.5}{\input{figs/fig3a-c.figtex}} \\
(a) \hspace{10mm} & (b)
\end{tabular}
{\caption{\label{xform3a}Transform 3a. Remove the transition with
    the size of postset of 1.}}
\end{center}
\end{figure}

\begin{figure}[tbh]
\begin{center}
\begin{tabular}{cc}
\scalebox{0.5}{\input{figs/fig3b-a.figtex}}
\scalebox{0.5}{\input{figs/fig3b-b.figtex}} \\
(a) \hspace{10mm} & (b)
\end{tabular}
{\caption{\label{xform3b}Transform 3b. The transformation in
    Figure~\ref{xform3a} without the restriction that places in preset(t)
    have postset sizes of 1.}}
\end{center}
\end{figure}

\begin{figure}[tbh]
\begin{center}
\begin{tabular}{cc}
\scalebox{0.5}{\input{figs/fig3c-a.figtex}}
\scalebox{0.5}{\input{figs/fig3c-b.figtex}} \\
(a) \hspace{10mm} & (b)
\end{tabular}
{\caption{\label{xform3c}Transform 3c.  The transformation in
    Figure~\ref{xform3a} without the restriction that the place in
    postset(t) have preset size of 1.}}
\end{center}
\end{figure}

Figure~\ref{xform3a}(a) shows a transition $t$ with only one place in its
postset.  Suppose $t$ is removed.  The incoming and outgoing flow relations
of $t$ are also removed.  Then the place in ${\tt postset}(t)$ is 
merged with those in ${\tt preset}(t)$.  The delay bounds of the transition 
$t$ are added to the delay bounds of the transitions following the places in
${\tt postset}(t)$, as shown in Figure~\ref{xform3a}(b).  
Figure~\ref{xform3a}(b) shows this transform applied when {e1}={e2}=true.
Figure~\ref{xform3a}(c) shows the same transform when {e2}$\Rightarrow${e1}
and {e3}$\Rightarrow${e1}.  If one or all of the transitions following $t$ has
an enabling condition that does not imply {e1}, then the transform cannot be
applied.

The transformation in Figure~\ref{xform3a}  has the restrictions that
for each place $p$ in
${\tt preset}(t)$, the size of ${\tt postset}(p)$ equals 1 and  for the
place $p$ in ${\tt postset}(t)$, the size of ${\tt preset}(p)$ equals
1.  Figures \ref{xform3b} and \ref{xform3c} show the same
transformations without each restriction respectively.

Figure~\ref{bad}(a) and (b) show where these above transformations
cannot be applied.  Namely, one or other of the above mentioned restrictions
may be loosened, but not both.

All places in preset(t) must have same marking (all marked or all
unmarked) and place in postset(t) can be marked.  If either preset(t)
or postset(t) is marked, new combined places are marked.

%\clearpage
%\newpage

\subsection{Transform 4: Remove A Transition With A Single Place in the Preset}
\label{xform-3}

\begin{figure}[tbh]
\begin{center}
\begin{tabular}{cc}
\scalebox{0.5}{\input{figs/fig4a-a.figtex}}
\scalebox{0.5}{\input{figs/fig4a-b.figtex}} \\
(a) \hspace{10mm} & (b)
\end{tabular}
{\caption{\label{xform4a}Transform 4a. Remove a transition with
    preset of size 1.}}
\end{center}
\end{figure}

\begin{figure}[tbh]
\begin{center}
\begin{tabular}{cc}
\scalebox{0.5}{\input{figs/fig4b-a.figtex}}
\scalebox{0.5}{\input{figs/fig4b-b.figtex}} \\
(a) \hspace{10mm} & (b)
\end{tabular}
{\caption{\label{xform4b}Transform 4b. Transformation in Figure~\ref{xform4a}
    without restriction that the places in preset(t) have a preset of
    size 0.}}
\end{center}
\end{figure}

\begin{figure}[tbh]
\begin{center}
\begin{tabular}{cc}
\scalebox{0.5}{\input{figs/fig4c-a.figtex}}
\scalebox{0.5}{\input{figs/fig4c-b.figtex}} \\
(a) \hspace{10mm} & (b)
\end{tabular}
{\caption{\label{xform4c}Transform 4c. Transformation in Figure~\ref{xform4a}
    without restriction that the places in postset(t) each have a preset of
    size 1.}}
\end{center}
\end{figure}

Figure~\ref{xform4a}(a) shows a transition $t$ with only one place in its
preset.  Suppose $t$ is removed.  The incoming and outgoing flow relations
of $t$ are also removed.  The the place in ${\tt preset}(t)$ is 
merged with those in ${\tt postset}(t)$.  The delay bounds of the transition 
$t$ are added to the delay bounds of the transitions following the places in 
${\tt postset}(t)$, as shown in Figure~\ref{xform4a}(b).
The transformation in Figure~\ref{xform4a} as the restrictions that
for the place $p$ in
${\tt preset}(t)$, the size of ${\tt postset}(p)$ equals 1 and for each
place $p$ in ${\tt postset}(t)$, the size of ${\tt preset}(p)$ equals
1, too.  Figures~\ref{xform4b} and \ref{xform4c} demonstrate the same
transform where these restrictions are loosened.

The above transformations are also 
restricted by the conditions shown in Figure~\ref{bad}(a) and (b).

All places in postset(t) must have same marking (all marked or all
unmarked) and place in preset(t) can be marked.  If anything is
marked, new combined places are marked.

This transform may only be meaningfully applied if the enabling conditions of
$t$ and all preceding transitions is true.

%\clearpage

\subsection{Transform 5: Merge Transitions With the same preset and postset}
\label{merge-1}

\begin{figure}[tbh]
\begin{center}
\begin{tabular}{cccc}
\scalebox{0.5}{\input{figs/fig5a-a.figtex}}
\scalebox{0.5}{\input{figs/fig5a-b.figtex}} \\
(a) \hspace{10mm} & (b)\\
\scalebox{0.5}{\input{figs/fig5a-c.figtex}}
\scalebox{0.5}{\input{figs/fig5a-d.figtex}} \\
(c) \hspace{10mm} & (d)
\end{tabular}
{\caption{\label{xform5a}Transform 5a. Merge transitions with the same
    preset and postset.  Figure (b) shows the transform if {e1}={e2}.
    Figures (c) and (d) show the transform if {e2}$\Rightarrow${e1} or
    {e1}$\Rightarrow${e2}, respectively.}}
\end{center}
\end{figure}

\begin{figure}[tbh]
\begin{center}
\begin{tabular}{cc}
\scalebox{0.5}{\input{figs/fig5b-a.figtex}}
\scalebox{0.5}{\input{figs/fig5b-b.figtex}} \\
(a) \hspace{10mm} & (b) 
\end{tabular}
{\caption{\label{xform5b}Transform 5b. Merge transitions where the
    places in their preset and postset have the same preset and
    postset.}}
\end{center}
\end{figure}

Figure~\ref{xform5a}(a) shows two transitions $t_6$ and $t_7$ with 
the same preset and postset.  Those two transitions can be merged together
without affecting the behavior on transitions $t_{10}$, $t_{11}$, and 
$t_{12}$ as shown in Figure~\ref{xform5a}(b).  Since no places are
being removed, no marking restrictions apply.  This transform cannot be
applied, however, if the assignments on the transitions are different.
Figure~\ref{xform5a}(b) shows the transform when {e1}={e2}.
Figure~\ref{xform5a}(c) and Figure~\ref{xform5a}(d) show the same transform
when {e2}$\Rightarrow${e1} and {e1}$\Rightarrow${e2}, respectively.

%Note: Transform 5b is not turned on by default.  It was found to have
%minimal effect except for in a few examples and was computationaly
%intensive.  Additionally, the correctness of this transformation in
%its current form is in question.

Another case is where the two transitions do not have the same preset
and postset, but the places in their preset and postset have the same
preset and postset as shown in Figure~\ref{xform5b}(a).  The two
transitions can still be merged along with the places in their preset
and postset. This transformation is depicted in Figure~\ref{xform5b}.
In order for the transformation to be applied certain timing
constraints must be satisfied.

%There is also requirement on this transformation that the places in
%the preset of the merged two transitions must have same postset
%excluding the merging transitions, and the places in the postset of
%the merged two transitions must have same preset excluding the merging
%transitions.  For example, in Figure~\ref{xform5b}(a), the places in
%the preset of $t_3$ and $t_4$ are $p_1$ and $p_2$.  The postset of
%$p_1$ and $p_2$ contains only $t_2$, excluding $t_3$ and $t_4$.  The
%places in the postset of $t_3$ and $t_4$ are $p_3$ and $p_4$.  The
%preset of $p_3$ and $p_4$ contains only $t_5$, excluding $t_3$ and
%$t_4$.

%\clearpage

\subsection{Transform 6: Merge Transitions With the same preset}
\label{merge-2}

\begin{figure}[tbh]
\begin{center}
\begin{tabular}{cc}
\scalebox{0.5}{\input{figs/fig6-a.figtex}}
\scalebox{0.5}{\input{figs/fig6-b.figtex}} \\
(a) \hspace{5mm} & (b)
\end{tabular}
{\caption{\label{xform6}Transform 6. Merge transitions with the same
    preset.}}
\end{center}
\end{figure}

Figure~\ref{xform6}(a) shows two transitions $t_1$ and $t_2$ with 
the same preset.  It is also required that the places in the postset of 
$t_1$ and $t_2$ must have the same preset aside from the two
transitions being merged.  Additionally, there can only be one place
in the postsets of t1 and t2.  Those two 
transitions can be merged together without affecting the behavior on 
transitions $t_3$, $t_4$, and $t_5$ as shown in Figure~\ref{xform6}(b).

Since the places in the postset of the merging transitions will be
merged, they must have a consistent marking.  The newly merged place
will have the same marking as the places being merged.
%\clearpage

\subsection{Transform 7: Merge Transitions With the same postset}
\label{merge-3}

\begin{figure}[tbh]
\begin{center}
\begin{tabular}{cc}
\scalebox{0.5}{\input{figs/fig7-a.figtex}}
\scalebox{0.5}{\input{figs/fig7-b.figtex}} \\
(a) \hspace{10mm} & (b)
\end{tabular}
{\caption{\label{xform7}Transform 7. Merge transitions with the same
    postset.}}
\end{center}
\end{figure}

Figure~\ref{xform7}(a) shows two transitions $t_1$ and $t_2$ with 
the same postset.  It is also required that the places in the preset of 
$t_1$ and $t_2$ must have the same postset aside from $t_1$ and $t_2$.
Those two 
transitions can be merged together without affecting the behavior on 
transitions $t_3$, $t_4$, and $t_5$ as shown in
Figure~\ref{xform7}(b).  Additionally, the transitions being merged
may each only have one place in their presets.  The places in the
preset of the transitions
being merged should have consistent markings.  The newly merged place
in the transformed net should have the same marking as the places that
are merged.



%\clearpage

\subsection{Illegal Transformations}

\begin{figure}[tbh]
\begin{center}
\begin{tabular}{cc}
\scalebox{0.5}{\input{figs/figbad-1.figtex}}
\scalebox{0.5}{\input{figs/figbad-2.figtex}} \\
(a) \hspace{6mm} & (b) \\
\scalebox{0.5}{\input{figs/figbad-3.figtex}}
\scalebox{0.5}{\input{figs/figbad-4.figtex}} \\
(c) \hspace{6mm} & (d)
\end{tabular}
{\caption{\label{bad}Illegal transformations.}}
\end{center}
\end{figure}

Figure~\ref{bad} shows several illegal transformations. In Figure~\ref{bad}(a),
neither $t_1$ nor $t_2$ can be removed using transformations in 
section~\ref{reduce1} and \ref{reduce2}.  
They may be merged by transformations insection~\ref{merge-1}, 
\ref{merge-2}, or \ref{merge-3}.  In Figure~\ref{bad}(b), $t_3$ cannot be
removed in that it is impossible to have two places connected without a
transition in between. Figure~\ref{bad}(c) shows a stand-alone self loop which
is not removed for some reason.  I think it can be removed if the wire of 
that transition is not used in any levels.  Figure~\ref{bad}(d) shows a
dangling transition $t_2$.  It cannot be removed because safety failure
maybe hidden; otherwise.

%\newpage

\section{Assignment Transforms}

This section presents transforms that build upon the transforms from the
previous section.  While these themselves do not simplify the LHPN (with the
exception of Transform 12), they remove assignments from certain transitions
so that Transforms 2-7 can then be applied to actually simplify the net.

\subsection{Transform 8: Combine Sequential Assignments to a Single Variable}
\label{reduce8}

\begin{figure}[tbh]
\begin{center}
\begin{tabular}{ccc}
\scalebox{0.5}{\input{figs/fig8-a.figtex}}
\scalebox{0.5}{\input{figs/fig8-b.figtex}}
\scalebox{0.5}{\input{figs/fig8-c.figtex}} \\
(a) \hspace{6mm} & (b) \hspace{6mm} & (c)
\end{tabular}
{\caption{\label{xform8}Transform 8. Sequential Assignments to a Single 
Variable.  Note: enab1$\Rightarrow$enab2.}}
\end{center}
\end{figure}

Figure~\ref{xform8}(a) shows two sequential transitions with assignments to a
single variable.  These assignments can be combined into a single assignment in
a two-step process.  First, each assignment is broadened to include the maximum
range included in the two assignments, as shown in Figure~\ref{xform8}(b).  In
each case, the assignments include a superset of the original values.  When this
has been done, the second assignment becomes vacuous and can be eliminated, as 
shown in Figure~\ref{xform8}(c).  This can be done if the assignments meet the
following criteria:
\begin{itemize}
\item The assignments are sequential.
\item Both assignments are to the same variable.
\item There are no concurrent assignments to the variable being assigned.
\item The second assignment has zero delay.
\item The enabling condition of the first transition implies the enabling
condition of the second assignment, i.e. enab1$\Rightarrow$enab2.
\end{itemize}

%\clearpage

\subsection{Transform 9: Combine Sequential Assignments to Multiple Variables}
\label{reduce9}

\begin{figure}[tbh]
\begin{center}
\begin{tabular}{cc}
\scalebox{0.5}{\input{figs/fig9-a.figtex}}
\scalebox{0.5}{\input{figs/fig9-b.figtex}} \\
(a) \hspace{6mm} & (b)
\end{tabular}
{\caption{\label{xform9}Transform 9. Sequential Assignments to Multiple 
Variables, if {e2}$\Rightarrow${e1}.}}
\end{center}
\end{figure}

Figure~\ref{xform9}(a) shows two sequential transitions, each with an assignment
to a different variable.  These assignments can be combined onto a single
transition, as seen in Figure~\ref{xform9}(b).  Care must be taken to apply this 
transform in the correct cases, because it will slightly alter the behavior of the
net.  In the case of Boolean variables, the state graph must have an intermediate
``unknown'' state in which each of the variables is unknown.  This allows the
variables to be assigned in any order, rather than the assignments being done 
simulataneously.  This transform can be done when:
\begin{itemize}
\item The assignments are sequential.
\item Both assignments are to different variables.
\item There are no assignments or enabling conditions that are dependent on the
variables involved.
\end{itemize}

%\clearpage

\subsection{Transform 10: Combine Parallel Assignments to Multiple Variables}
\label{reduce10}

\begin{figure}[tbh]
\begin{center}
\begin{tabular}{cc}
\scalebox{0.5}{\input{figs/fig10-a.figtex}}
\scalebox{0.5}{\input{figs/fig10-b.figtex}} \\
(a) \hspace{10mm} & (b)
\end{tabular}
{\caption{\label{xform10}Transform 10. Parallel Assignments to Multiple 
Variables.}}
\end{center}
\end{figure}

Figure~\ref{xform10}(a) shows two parallel transitions, each with assignments to a
different variable.  These assignments can be put on the same transition as shown
in Figure~\ref{xform10}(b).  Care must be taken to apply this transform in the 
correct cases, because it will slightly alter the behavior of the net.  In the 
case of Boolean variables, the state graph must have an intermediate ``unknown'' 
state in which each of the variables is unknown.  This allows the variables to be
assigned in any order, rather than the assignments being done simulataneously.  
This transform can be done when:
\begin{itemize}
\item The assignments are parallel.
\item Both assignments are to different variables.
\item There are no assignments or enabling conditions that are dependent on the
variables involved.
\end{itemize}

%\clearpage

\subsection{Transform 11: Simplify Cascading Assignments}
\label{reduce11}

\begin{figure}[tbh]
\begin{center}
\scalebox{0.5}{\input{figs/fig11-a.figtex}}
\scalebox{0.5}{\input{figs/fig11-b.figtex}}
{\caption{\label{xform11}Transform 11. Cascading Assignments.}}
\end{center}
\end{figure}

Figure~\ref{xform11}(a) shows two transitions with assignments to different
variables.  If the variable in the first assignment is a statically determinable
(i.e. the variable is an integer or Boolean value or the rate of the variable is 
zero), then the value of the first variable can be inserted into the second
assignment in place of the variable name.  If there are no more assignments or 
enabling conditions that depend on the value of the first variable before it is 
assigned a new value, the assignment to that variable may be removed.  Otherwise,
it can be moved onto the same transition as the other variable, as shown in
Figure~\ref{xform11}(b).  Care must be taken to apply this transform in the 
correct cases, because it will slightly alter the behavior of the net.  In the 
case of Boolean variables, the state graph must have an intermediate ``unknown'' 
state in which each of the variables is unknown.  This allows the variables to be
assigned in any order, rather than the assignments being done simulataneously.  
This transform can be done when:
\begin{itemize}
\item The assignments are sequential.
\item The value of the first variable is statically determinable.
\end{itemize}

%\clearpage
%\pagebreak

\subsection{Transform 12: Simplify Assignment Loop}
\label{reduce12}

\begin{figure}[tbh]
\begin{center}
\scalebox{0.5}{\input{figs/fig12a-a.figtex}}
\scalebox{0.5}{\input{figs/fig12a-b.figtex}} \\
(a) \hspace{50mm} (b) \\
\scalebox{0.5}{\input{figs/fig12a-c.figtex}}
\scalebox{0.5}{\input{figs/fig12a-d.figtex}} \\
(c) \hspace{50mm} (d)
{\caption{\label{xform12}Transform 12. Assignment Loop.}}
\end{center}
\end{figure}

Figure~\ref{xform12}(a) shows an assignment loop that is frequently seen in LHPNs
generated by SPICE data.  The assignments and enabling conditions can be shifted,
removed, or combined as shown in Figure~\ref{xform12}(b).  The empty transitions
can then be removed to form the subnet shown in Figure~\ref{xform12}(c).  This 
transorm can be done when:
\begin{itemize}
\item This specific structure of places and transitions exists.
\item For the enabling conditions as shown in Figure~\ref{xform12}, {\tt a$\Rightarrow$b} and {\tt d$\Rightarrow$c}.
\end{itemize}

\subsection{To Do and Notes}

\begin{itemize}
\item Elaborate on the uncertainty of combining variables with expressions.
\end{itemize}

\end{document}