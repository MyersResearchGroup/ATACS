%%%%%%%%%%%%%%%%%%%%%%%%%%%%%%%%%%%%%%%%%%%%%%%%%%%%%%%%%%%%%%%%%%%%%%%%%%%%%%%
%% Copyright (c) 2002 Scott Little <little@eng.utah.edu>
%%
%% File Name     : developer.tex
%% Author        : Scott R. Little
%% E-mail        : little@eng.utah.edu
%% Date Created  : 11/01/2002
%%
%% Description   : A document providing useful information for
%%                 ATACS developers.
%%
%% ToDo          : Add some sort of index.
%%
%%%%%%%%%%%%%%%%%%%%%%%%%%%%%%%%%%%%%%%%%%%%%%%%%%%%%%%%%%%%%%%%%%%%%%%%%%%%%%%

\documentclass[titlepage,11pt]{article}
\textwidth 6.5in
\textheight 9in
\oddsidemargin -0.2in
\topmargin -0.5in
\usepackage{indentfirst,graphics,alltt,epsfig,color}

\title{The ATACS Developer's Manual}
\date{Created: November 1, 2002\\
  Last Revised: February 5, 2007
}

\begin{document}
  \maketitle

  %show only subsection granularity in the toc
  \setcounter{tocdepth}{2} 

  \tableofcontents \newpage
  
   \setlength{\parindent}{0em} \setlength{\parskip}{10pt}
  
   \section{Introduction \& Purpose}
   This document intends to give a basic overview of the core concepts
  required for ATACS development. It describes how to build ATACS, the
  necessary libraries, and some optional tools. It also covers the basic
  operation of the source management and profiling infrastructure provided for
  ATACS developers. The basic ATACS data structures and core functionality are
  also briefly defined and described. For more information about basic
  operation and use of ATACS please see the ATACS User's Manual.

  \section{Building ATACS on Linux}
  Compiling ATACS for the first time on a new system may be a bit
  daunting at first, but it really isn't that hard.  The initial setup
  takes a little bit of time, but once the system is setup correctly it
  is very easy to maintain a working build.  There are a few required
  libraries and helpful tools that ATACS uses.  These are described in
  the remaining portions of this section.
  
  \subsection{Required libraries}
  This section briefly describes the required libraries and how to
  build them.  This is just meant as a quick help.  As always you
  should consult the documentation that comes with the libraries for
  a more complete set of instructions.
  
  \subsubsection{cprob}
  The cprob libraries are really quite simple to build. They will compile
without problem using almost any compiler. The source code can be obtained
from: http://scicomp.ewha.ac.kr/netlib/cephes/ (Note: the version of this
library available on the web is incompatible with ATACS.). To build the
library you simply untar the cprob.tgz and do a 'make -f cprob.mak'. You then
need to install the library (libprob.a) into a location that is in the library
path. /usr/lib or /usr/local/lib are both nice locations depending on your
preference. If you do put the libraries in /usr/local/lib you will be required
to modify the ATACS Makefile by adding /usr/local/lib to the library path.
  
  \subsubsection{cudd}
  Information about cudd can be found at
  http://vlsi.colorado.edu/\verb1~1fabio/CUDD/cuddIntro.html  Cudd
  can be built by just typing 'make'.  The include directory should be
  copied to some useful include location.  /usr/include or
  /usr/local/include are acceptable locations.  Then you should go
  through the subdirectories to find the libraries that were built.
  You should again put them in a useful location such as /usr/lib or
  /usr/local/lib.  You should find: libcudd.a, libdddmp.a, libepd.a,
  limtr.a, libobj.a, libst.a, and libutil.a.  You should rename
  libutil.a to libCutil.a for use with ATACS.  If you put the include
  files or library files in /usr/local then small modifications to the
  ATACS Makefile will be required.
  
  \subsection{Required tools}
  The tools required to build ATACS are common tools that you would
  expect to find on most Linux systems.  The required minimum set of
  tools is: g++, flex, bison, GNU make, sed, and Perl.
  
  \subsection{Useful tools}
  These are a couple of tools that ATACS uses to display state graphs,
  error traces, etc.  These tools certainly aren't required, but it is
  nice to have the ability to receive visual feedback about results or
  errors.
  
  \subsubsection{dot (graphviz tools)}
  Dot is a part of the graphviz tool suite that can be found at:
  http://www.research.att.com/sw/tools/graphviz/.  Dot is used to
  display graphical representations of the circuits that ATACS is
  synthesizing or verifying.  It may not always be wanted, but can be
  a very useful debugging tool in many instances.  Building dot is
  quite simple.  The instructions provided with the source code should
  be sufficient to build dot successfully or you can simply download a
  binary package for you system if one is available.
  
  \subsubsection{parg}
  Parg is another tool used to display graphical representations of
  the circuits.  The parg source code can be found in the atacs/src
  directory and is build by typing 'make parg' from the atacs/src
  directory.  

  \subsubsection{dinotrace}
  Dinotrace is a simple waveform viewer that is used by ATACS to
  display the trace that causes a particular error.  The source code
  can be found at: http://www.veripool.com/dinotrace/.  The build
  instructions are included with the source code and are standard
  instructions.  Simply build and install dinotrace to start seeing
  the error traces generated by ATACS.
  
  \subsection{Building ATACS on a non-Linux system}

  \subsubsection{Cygwin \& Windows}
  ATACS has successfully been built under Windows using the Cygwin
  tool suite.  The Cygwin tool suite can be obtained at:
  http://www.cygwin.com/.

  **Add required changes to build ATACS under Cygwin.
  
  \subsubsection{HP-UX}
  In the past ATACS has been compiled under HP-UX.  This has not
  happened for a little while.  The build may be broken or it may
  still work.  That is currently unknown.  That said it is very
  possible to get ATACS running on HP-UX with a little bit of work.

  \section{Environment variables, CVS, and Make}
  Okay, if you have managed to follow the directions up to this point
  or are already working on a machine that has the proper libraries
  and software installed then you are ready to actually get the ATACS
  source code, build it, and run ATACS.  The remainder of this section
  should help you do just that.
  
  \subsection{Checking out ATACS}
  The ATACS source code is stored in a CVS repository in
  /home/ming/cvsroot on ming.ece.utah.edu.  To check out a copy of
  ATACS from a machine in the async lab you can simply type 'cvs -d
  /home/ming/cvsroot co atacs' if you are located on a machine that
  has an NFS mount of ming's file system.

  If you are not located on a machine that has an NFS mount of ming's
  file systems then the process is slightly different.  You will need
  to set your CVS\_RSH environment variable to the value 'ssh'.  You
  can then do a 'cvs -d ming.ece.utah.edu:/home/ming/cvsroot checkout
  atacs'.

  After typing the appropriate cvs command, you should see a large
  amount of text scroll across the screen and upon termination there
  should now be an atacs directory in your current working directory.
  This directory contains all of the ATACS source code.
  
  \subsection{Building ATACS}
  To build atacs you need to enter the atacs/src directory and simply
  type make.  If everything is setup correctly then ATACS should
  finish building without complaint.
  
  \subsection{ATACS Environment variables}


  \section{Basic ATACS CVS and Make}

  \subsection{Making changes}

  \subsection{Testing your changes}

  \subsection{Committing changes}

  \subsubsection{Local CVS vs. Remote CVS}  
  
  \section{Advanced ATACS CVS and Make}
  We use both CVS and Make extensively in the development of ATACS.
  The majority of this infrastructure was put in place by Eric
  Peskin.  Parts of these files are well documented.  Below is an
  e-mail communication that I had with Eric.  It sums up a lot of the
  information and will be placed here until I have more time to edit
  and organize it.

  \subsection{Where are the critical CVS scripts located?  What is the
    main job of each script?}
  
  It all begins in the CVS module called CVSROOT.  To begin working on
  anything in this module, you should first do cvs checkout CVSROOT
  somewhere, and so on...)
  
  \textbf{CVSROOT/commitinfo}\\
  This is the first thing that CVS consults before a commit.  If the
  commit involves anything in atacs/{examples,src}, then control is sent
  to /home/ming/remote/atacs/examples/regress (see below).
  
  \textbf{CVSROOT/loginfo}\\
  CVS consults this AFTER a commit has completed.  I've set this up such
  that after each local commit, it calls
  /home/ming/remote/atacs/examples/refresh (see below).  This makes sure
  that the tree at /home/ming/remote/atacs stays current
  
  
  Then there are the scripts that are managed in the atacs cvs module:
  Note: The executable scripts under atacs/examples should probably be
  moved to atacs/bin instead (because they are executable, and they
  are not actually examples).  They are in examples for historical
  reasons: they are closely associated with the examples/Makefile, and
  we didn't used to have an atacs/bin directory.
  
  \textbf{atacs/examples/regress}  \emph{perl script}\\
  This script determines whether it is a local commit or a remote
  commit.  For a remote commit it copies the files to be committed on
  top of the image at /home/ming/remote/atacs, and runs al the tests
  there.
  
  For a local commit, CVS may will call regress multiple times, once for
  each subdirectory containing stuff to be committed.  So for a local
  commit, the job of regress is to call status.perl (see below) exactly
  once for each cvs commit, passing complete information about where and
  how cvs was called.
  
  
  \textbf{atacs/examples/status.perl}
  The job of this script is to warn the user if the corresponding cvs
  commit would be likely to break the repository.  In particular, it
  checks to see whether the user has modified other files that are not
  being committed.  If there are no warnings -- or if the user chooses
  to ignore any warnings -- this script calls make on
  atacs/examples/Makefile to run the regression tests
  
  \textbf{atacs/examples/Makefile}
  This Makefile handles the actual regression tests.  It first calls
  make on atacs/src/Makefile to make sure the atacs executable is up to
  date.  Then it runs the regression tests on that example.

  \textbf{atacs/examples/refresh} \emph{shell script}
  This brings the tree at /home/ming/remote/atacs up-to-date.
  (Actually, it brings whatever tree it was called in up-to-date, but I
  only call it in /home/ming/remote/atacs.  See CVSROOT/loginfo.)  It
  removes any files with conflicts, does an update, and then fixes
  permissions.
  
  \textbf{atacs/examples/certify} \emph{shell script}
  This script drives the nighttime regression testing.  At its heart, it
  just calls make -k Certify on the atacs/examples/Makefile.  But it
  also does the automatic tagging of the tested version, and
  automatically submits a bug report if something goes wrong.
  
  
  \textbf{atacs/src/Makefile}
  Builds the atacs executable from the atacs src
  
  
  \textbf{\large{Then there's a script that's not CVSed at all (!):}}\\
  \textbf{ming:/etc/cron.daily/certifyATACS}\\
  This runs once a night, updates the atacs tree at
  \verb1~1peskin/nobackup/cert, and then, if anything has changed since the
  last time, it runs atacs/examples/certify there.
  
  Finally, there are some directories to know about:
  /home/ming/remote/atacs/ is the testbed for remote commits.
  \verb1~1peskin/nobackup/cert/ is the testbed for nightly regression testing.
  \verb1~1atacs-bugs is the directory for the jitterbug database (see below)
  
  
  \subsection{How can I create custom example sets for the examples Makefile?}
 
  1) Make such each example in your set is checked in as
  atacs/examples/\verb1<1type\verb1>1/\verb1<1example\verb1>1.\verb1<1type\verb1>1\\
  2)  Create your own suffix instead of .gold, say, .\verb1<1scottsuffix\verb1>1\\
  3)  Check in each prs file of expected results as
  atacs/examples/\verb1<1type\verb1>1/\verb1<1example\verb1>1.prs.\verb1<1scottsuffix\verb1>1\\
  4)  In the atacs/examples directory, type
  make Expected=\verb1<1scottsuffix\verb1>1\\
  That will run your example set with the standard flags.  To alter the
  flags, the most common approach is to use the Extras variable to add
  additional flags:
  make Expected=\verb1<1scottsuffix\verb1>1 Extras=\verb1<1scottsextraflags\verb1>1
  There are many other variables you can customize.  See the comments in
  atacs/examples/Makefile\\
  5)  If you do this often, you could add a target to the Makefile like

  Scott:
  \${MAKE} Expected=\${scottsuffix} Extras=\${scottsextraflags}
  
  For example, see the rule for the target hairyNoPostProc in
  atacs/examples/Makefile
  
  6)  Then, in the future, you could just type
  make Scott
  
  \subsection{Are there more Makefiles than the main Makefile and the
  examples Makefile?}
  
  Just papers/LaTeX/Makefile
  
  \subsection{Are the scripts for remote checkin different?}
  
  Yes, some of them are, but I've included them above.
  
  First to sum up:\\
  For a LOCAL commit the call sequence of scripts is:\\
  CVSROOT/commitinfo\\
  atacs/examples/regress\\
  atacs/examples/status.perl\\
  atacs/examples/Makefile\\
  (tests run in user's local directory)\\
  atacs/examples/refresh runs after the commit to update
  /home/ming/remote/atacs/
  
  For a REMOTE commit the call sequence is:\\
  CVSROOT/commitinfo\\
  atacs/examples/regress\\
  atacs/examples/Makefile\\
  (tests run in /home/ming/remote/atacs)
  
  For nighttime regression the call sequence is:\\
  ming:/etc/cron.daily/certifyATACS\\
  atacs/examples/certify\\
  atacs/examples/Makefile\\
  (tests run in \verb1~1peskin/nobackup/cert/atacs/)
  
  \subsection{Is remote checkin administered differently? If so, how do I do
  the admin in case of a problem?}
  
  The admin consists of administering the above scripts plus making sure
  the /home/ming/remote/atacs tree stays in a good state.  Despite the
  best efforts of the atacs/examples/refresh script, sometimes the tree
  at /home/ming/remote/atacs somehow gets wedged in a weird state.  So
  sometimes I have to manually remove the files with conflicts, manually
  update, and manually fix permissions.  Or sometimes I just blow away
  the whole tree and do a fresh checkout and make, and then fix
  permissions.  (The whole tree needs to be group async.  On each file,
  group must have all the permissions that the owner does.  Non-group
  people should have no permissions.)
  
  \subsection{What are the current known holes in the checkin process?}
  
  OK.  So one hole is whatever causes the automation to fail, requiring
  the manual intervention I describe above.
  
  Another hole is the fact that a naughty developer could change files
  around while emacs is waiting for a log message.  For that matter a
  naughty developer could turn off the whole thing by changing
  CVSROOT/commitinfo to just allow any commit.  Hmmm.
  
  Another hole is the warning mechanism during local commits that we
  discussed during my practice talk.
  
  Within the regression testing, one big hole is that I never simulate
  the circuits to see if they actually do what they are supposed to.  I
  just check that atacs gets the same answer it did last time.  But we
  have VHDL as an input form.  And storevhd and tel2vhd attempt to
  provide VHDL as an output form.  So some simulation should be within
  reach.
  
  Also, for the most part, we just check one combination of atacs
  switches.  It's an end-to-end path through the code, but it's just one
  path.  We should test more.
  
  \subsection{What would you add to the checkin process if you had
    infinite time?}
  
  Plugging the holes above is one thing.\\
  Making local commits as robust as remote commits .... or just
  convincing everyone to always use remote commits even when they are
  really local.
  Simulation.\\
  Test greater variety of switches and paths through code.\\
  Make test harness easier for others to extend.\\
  Link bug tracking database to CVS.\\
  Automatically associate each message with tag.\\
  Educate!\\
  Document!
  
  \subsection{Where are the jitterbug files?}
  \begin{itemize}
  \item The executables are /home/shang/httpd/cgi-bin/atacs*
    \subitem These must have permission set as follows:
    \begin{verbatim}
        chown root.www atacs atacs.private
        chmod 04710 atacs atacs.private
    \end{verbatim}
  \item The configuration files are in shang:/etc/jitterbug/
    AND its mirror at chou:/etc/jitterbug/
  \subitem On both machines, /etc/jitterbug/atacs needs a line that
    points to the IP address of the mail server.  E.g., 
      smtp address = 155.98.27.202.
    If the IP address of the mail server changes, you'll need to edit
    this line on both machines.
  \item The atacs-bugs home directory is \verb1~1atacs-bugs/
  \subitem   \verb1~1atacs-bugs/.procmailrc pipes mail to the new\_message
  \subitem \verb1~1atacs-bugs/bin/new\_message
    There needs to be a link to new\_message in chou:/etc/smrsh/
    See items 39e,f in our RedHat/RedHat\_7.1\_upgrade.txt
  \subitem The actual bug reports and html files are stored under
  \verb1~1atacs-bugs/bug\_tracking/  This directory should be
  world-readable (but only writable by atacs-bugs).
  \item The source code is at /home/chen/usr/local/src/jitterbug/
  \end{itemize}

  \subsection{Where can I find jitterbug documentation? Google says: 
  http://samba.anu.edu.au/jitterbug/ ... is that correct?}

  Yes.  See also /home/chen/usr/local/src/jitterbug/docs/

  \subsection{Extended information on the examples/Makefile internal
    workings}
  
  \textbf{Question:} I have been trying to use the examples Makefile
  to test my code, but it just isn't seeming to work very well.  I
  keep getting flags that I don't want and I can't figure out how to
  turn them off.  I have stripped all of the "offending" flags out
  of the Makefile.  Well, actually out of my own custom file.  I
  made a makefile.srl so as not to corrupt the main Makefile.  Maybe
  this is the problem.  The command that I am using is:
  
  make -f makefile.srl srl hse/PRSs.subDir
  
  the srl target is as follows:
  srl:
  \${MAKE} Extras=tSoOofoqlhsgllva
  
  This is the command that the Makefile runs: ../../bin/atacs
  abstract.csp -tpG0-1tSoOofoqlhsgllvays \verb1>1 abstract.log 2\verb1>1\&1
  
  Where do the -tpG0-1 and ys flags come from?  They are no
  longer anywhere to be found in makefile.srl b/c I ripped them
  out in a fit of frustration.  Any help would be appreciated.  I
  just can't seem to parse the Makefile well enough to see where
  it is getting these flags.

  \textbf{Answer from Chris:} Not sure what is going on here, but with
  the original make file, you can do this:
  
  make --keep-going Timing=tSoOofoq DoIt=lhsgllva hse/check.subDir
  
  The --keep-going tells it to keep going even if the results don't match 
  (they won't match because you aren't producing logic).
  
  Timing is used for timing flags.
  
  DoIt is used for command flags.
  
  \textbf{Answer from Eric Peskin:} The problem here is that as the
  Makefile recurses into subdirectories, it specifically calls
  atacs/examples/Makefile.  This is controlled by the rule:
  
  \# Take any pseudo-target of the form \verb1<1directory\verb1>1/\verb1<1target\verb1>1.subDir to mean:\\
  \%.subDir:       \# enter \verb1<1directory\verb1>1, and then make \verb1<1target\verb1>1\\
  \${MAKE} -C \$(@D) -f \${Root}/Makefile Root=../\${Root} Dir=\${Dir}/\$(@D) \$(*F)
  
  In particular, note the "-f \${Root}/Makefile" above
  
  I bet the result is that you're really only using your Makefile.srl
  for the top level and then all the real work is happening in my
  Makefile.  To let Makefile.srl take over the whole process you'd need
  to change the above rule in Makefile.srl to use
  -f \${Root}/Makefile.srl
  instead of just
  -f \${Root}/Makefile
  
  BUT, I like Chris' solution even better.  As he points out, most
  things like this can be done from the command line without changing
  the Makefile at all.  Sorry I neglected to mention the Timing and DoIt
  variables in my earlier e-mail.
  
  A word about --keep-going:  This causes make to barrel ahead no matter
  what happens: even if verification fails, or there's a segmentation
  fault or whatever.  So if you use this, you'll want to look carefully
  at any error messages.  (The idea of --keep-going is to run as much as
  possible in batch mode and look over the report later.)
  
  A couple of alternatives:  You could use 
  make Diff=/bin/true Timing=tSoOofoq DoIt=lhsgllva hse/check.subDir 
  This swaps in the dummy program /bin/true (which always returns 0)
  instead of compare\_literals.  So the above command will still grind to
  a halt if atacs crashes or returns nonzero exit status, but it won't
  try to compare results.
  
  But wait -- it looks like what you're really trying to do is to see
  whether things verify.  I actually already added support for this.
  See the targets verificationTests, verify, and .\%\_VERIFY\${ATACSflags}.
  
  Basically, you use verify instead of check.  So...
  
  MY FAVORITE SOLUTION:
  1) Make up a new suffix, say .srl\\
  2) For each atacs/examples/hse/\verb1<1example\verb1>1.hse you want
  to verify: touch atacs/examples/hse/\verb1<1example\verb1>1.srl\\
  3) cd atacs/examples\\
  4) make Expected=srl Timing=tSoOofoq DoIt=lhsgllva hse/verify.subDir\\
  5) If this is going to be a regular thing:\\
  cvs add hse/*.srl\\
  cd .. \# to the main atacs directory\\
  cvs commit\\
  6) If it's going to be really regular:\\
  Add your own target to the atacs/examples/Makefile:\\
  SRL:\\
  \${MAKE} Expected=srl Timing=tSoOof\${noparg} DoIt=lhsgll\${verify} hse/verify.subDir
  
  Note that only the optional step 6 modifies the Makefile.
  
  Also note: As far as I know, the options oOofoq have nothing to do
  with timing.  So it might make more sense to use the Special
  variable as in make Expected=srl Timing=tS Special=oOofoq
  DoIt=lhsgllva hse/verify.subDir
  
  However, in this case, it shouldn't make a difference.

  \subsection{Shell limits}
  It is often useful to set limits in your shell to prevent example that cause
the disk to swap from running on a make in the examples directory. Shell
limits can be set in the following way if you are using tcsh. In bash the
command is ulimit.

  Here is the excerpt from the tcsh man page:
   limit [-h] [resource [maximum-use]]
         Limits the consumption by the current process  and
         each process it creates to not individually exceed
         maximum-use on the specified resource.  If no max-
         imum-use  is  given,  then  the  current  limit is
         printed; if no resource is given, then all limita-
         tions  are  given.   If  the -h flag is given, the
         hard limits are used instead of the  current  lim-
         its.  The hard limits impose a ceiling on the val-
         ues of the current limits.   Only  the  super-user
         may raise the hard limits, but a user may lower or
         raise the current limits within the legal range.

         Controllable resources currently  include  cputime
         (the  maximum  number of cpu-seconds to be used by
         each process), filesize (the largest  single  file
         which  can  be  created),  datasize  (the  maximum
         growth of the data+stack region via sbrk(2) beyond
         the end of the program text), stacksize (the maxi-
         mum  size  of  the  automatically-extended   stack
         region),  coredumpsize  (the  size  of the largest
         core dump that will be  created),  and  memoryuse,
         the  maximum  amount  of physical memory a process
         may have allocated to it at a given time.

         maximum-use may be given as a (floating  point  or
         integer)  number  followed by a scale factor.  For
         all limits other than cputime the default scale is
         `k' or `kilobytes' (1024 bytes); a scale factor of
         `m' or `megabytes' may also be used.  For  cputime
         the  default  scaling  is `seconds', while `m' for
         minutes or `h' for hours, or a time  of  the  form
         `mm:ss' giving minutes and seconds may be used.

         For  both  resource names and scale factors, unam-
         biguous prefixes of the names suffice.

  \section{Adding a flag to ATACS}
  As an ATACS developer you will certainly be interacting with command
  line flags on a regular basis.  One of the things that you will
  probably want to do is add your own command line flag.  This isn't
  exactly the easiest thing to do.  Hopefully this section will give
  you enough of an idea about how it is done so that you can add your
  own flags as necessary.

  One of the first steps is to add your boolean variable to store
  whether the flag is on or off to struct.h.  You can look at other
  examples of flags to see how this is done.

  Then you need to edit memaloc.c and set the default state for your
  newly added variable.

  Now def.h needs to be opened so you can add the one letter
  abbreviation for your new flag that will actually be used on the
  command line.
  
  help.c is the next file that should be changed.  Here you need to
  increment the command count and then add your method to the
  command$\_$list array in the appropriate location based on alphabetical
  order.  Information should also be added to the printman and/or
  printman$\_$short functions as applicable.
  
  Finally atacs.c should be editted.  Again this is dependent on what
  type of flag you are adding.  The idea that has been successful for
  me is to find a similar flag and follow the method used by that
  flag.  That should do it.  Run make and find something else to do
  for a bit because the recompile will take a bit.

  \section{Data Structures}
  An important part of ATACS development is understanding the data
  structures.  This section will give a basic description of the major
  ATACS data structures contained in struct.h.  Each section below
  describes a different data structure.  The section begins with the
  actual data structure taken from struct.h.  The remainder of the
  section describes the different parts of the data structure.

  \subsection{markingADT}
  \begin{verbatim}
  typedef struct marking_struct {
    char * marking;
    char * enablings;
    char * state;
    char * up;
    char * down;
    unsigned int ref_count;
  } *markingADT;
  \end{verbatim}
  
  The markingADT is used to hold the information to describe the
  current marking of the graph.

  \subsubsection{marking}
  Marking is a char* that is nrules long.  The valid values for
  marking are 'T' or 'F'.  The rule is marked either true or false
  depending on whether or not it is marked in the current state.
  
  \subsubsection{enablings}
  Enablings is a char* that is nevents long.  The valid values for
  enablings are 'T' or 'F'.  The events is set to true or false based
  on the whether or not the event is enabled in the current state.
  
  \subsubsection{state}
  State is a char* that is nsignals long.  The valid values for state
  are 0, 1, 'R', and 'F'.  These values represent the state of the
  wire (signal) in the current state where 0 stands for low, 1 for
  high, 'R' for rising, and 'F' for falling.
  
  \subsubsection{up, down, ref\_count}
  These are not applicable to most developers.
  
	\subsection{lhpnMarkingADT}
	\begin{verbatim}
		typedef struct lhpnMarkingStruct 
		{
		  char* marking;
		  char* enablings;
		  char* state;
		} *lhpnMarkingADT;
	\end{verbatim}
	
	\subsubsection{marking}
	An array that is nrules long that gives the status of the rule.  'T' if the rule is marked (there is a token in the discrete place of the preset).  'F' if the rule is not marked (there is no token in the discrete place of the preset).
	
	\subsubsection{enablings}
	An array that is nevents long that is true ('T') if the transition is enabled and false ('F') otherwise.
	
	\subsubsection{state}
	An array nsignals long that shows the state of each signal: '1' (the signal is true), '0' (the signal is false), 'R' (the signal is rising), or 'F' (the signal is falling).

  \subsection{markkeyADT}
  \begin{verbatim}
    typedef struct markkey_struct {
      int enabling;
      int enabled;
      int epsilon;
    } *markkeyADT;
  \end{verbatim}

  The markkey really serves as an index into the rules matrix and a
  cache for other useful information.  It is nrules long so that the
  information provided by markkey\_struct is available for each rule.  
  
  \subsubsection{enabling}
  Enabling is an integer index into the rules matrix.  This is the
  index into the row and represents the event or place that has
  enabled this rule.
  
  \subsubsection{enabled}
  Enabled is an integer index into the rules matrix.  This is the
  index into the column and represents the event or place that this
  rule enables.
  
  \subsubsection{epsilon}
  Epsilon in an integer values that should be 0 or 1.  It represents
  whether the rule is initially marked or not with 1 meaning the rule
  is initially marked and 0 meaning that it is not.  This is a
  duplicate of information contained in the ruleADT.
  
  \subsection{ruleADT}
  \begin{verbatim}
    typedef struct rule_struct {
      bool concur;
      bool markedConcur;
      int epsilon;
      int lower;
      int upper;
      int weight;
      int plower;
      int pupper;
      int predtype;
      int ruletype; 
      int marking;
      int maximum_sep;
      bool reachable;
      int conflict;
      level_exp level;
      level_exp POS;
      char * expr;
      CPdf *dist;
      double rate;
			ineqADT inequalities;
    } *ruleADT;
  \end{verbatim}
  
  The ruleADT is a 2D sparse matrix.  It is (nevents+nplaces) long by
  (nevents+nplaces) wide.  Each rule has an entry with all of the
  information contained in the rule\_struct.  Finding all the valid
  rules in a row provides all outgoing transitions from an event or
  place (the postset) while finding all valid rules in a column will
  provide all incoming transitions to that event or place (the
  preset).
  
  Another point of interest may be that often times when looking at
  code that is searching through the rules matrix the search will
  begin with event 1 instead of event 0.  This is because for
  historical reasons event 0 is always the reset event.  The reset
  event can generally be ignored, but occasionally the reset event is
  used for a placeholder or something similar.

  \subsubsection{epsilon}
  Epsilon in an integer value that should be 0 or 1.  It represents
  whether the rule is initially marked or not with 1 meaning the rule
  is initially marked and 0 meaning that it is not.  This information
  is copied into the markkeyADT.
  
  \subsubsection{lower, upper}
  Lower and upper and integer values that represent the lower and
  upper timing bounds on the rule.

  \subsubsection{weight}
  The weight assigned the rule, i.e. how many tokens are removed from
  the preset and added to the postset when the rule fires.  This is
  used in the HPN code.

  \subsubsection{plower, pupper}
  The lower and upper bounds for the predicates assigned to the rule.
  This is used in the HPN code.

  \subsubsection{ruletype}
  Ruletype is an integer value representing the type of rule.  The
  ruletypes are defined in struct.h and given below.

  \begin{verbatim}
    #define   NORULE       0
    #define   SPECIFIED    1
    #define   DISABLING    2
    #define   ASSUMPTION   4
    #define   TRIGGER      8
    #define   PERSISTENCE  16
    #define   ORDERING     32
    #define   NONPERS      64
    #define   CONTEXT      128
    #define   REDUNDANT    256
  \end{verbatim}
  
  \subsubsection{marking}
  Marking is an integer index into the markingADT or markkeyADT.  I
  like to think of this as a unique integer identifier for each rule.
  This identifier can be used to obtain information about that rule
  from the markingADT or markkeyADT.  For example to determine if rule
  i is marked you can examine marking-\verb1>1marking[i].  Or to find
  out which event rule i has enabled you can examine
  markkey[i]-\verb1>1enabled.
  
  \subsubsection{conflict}
  Conflict is an integer value representing the conflict type.  The
  conflict types are defined is struct.h and given below.

  \begin{verbatim}
    #define   NOCONFLICT   0
    #define   ICONFLICT    1
    #define   OCONFLICT    2
  \end{verbatim}
  
  \subsubsection{level}
  Level is of type level\_exp and is the SOP form of the boolean
  expression that may be present on the rule.
  
  \subsubsection{POS}
  Level is of type level\_exp and is the POS form of the boolean
  expression that may be present on the rule.

  \subsubsection{inequalities}
	Inequalities is the list of inequalities on the rule.

  \subsubsection{concur, markedConcur, maximum\_sep, expr, dist, rate}
  Not used by most developers.
  
  \subsection{eventADT}
  \begin{verbatim}
    typedef struct event_struct {
      char * event;
      bool dropped;
      bool immediate;
      int color;
      int signal;
      int lower;
      int upper;
      int process;
      int type;
      char * data;
      char * hsl;
      int rate;
      int lrate;
      int urate;
      int lrange;
      int urange;
      int linitrate;
      int uinitrate;
      level_exp SOP;
      ineqADT inequalities;
    } *eventADT;
  \end{verbatim}

  The eventADT is (nevents+nplaces) long and contains an event\_struct
  for each of these entries.

  \subsubsection{event}
  A char* that gives the symbolic name for an event.  These entries
  are of the form a+/1, b+/1, b-/1, \$c+/1, etc.  The \$ in front of an
  event indicates that the event is a sequencing event (i.e. added by
  ATACS and not explicitly described in the circuit specification).  
  
  \subsubsection{dropped}
  A boolean value to determine whether this event should be ignored.
  If dropped in true then this event is no longer applicable and can
  safely be ignored.
  
  \subsubsection{color}
  This is currently a free integer variable without a specified use.
  
  \subsubsection{signal}
  This is an integer index into the signalADT.  A value of -1
  means that this event does not have an index into the signalADT.  

	\subsubsection{lower/upper}
	The lower/upper timing bound for the event (transition).

  \subsubsection{type}
  This is an integer type describing the type of event.  The types are defined in ly.h.  A complete list is found below.  CONT is set when the event is continuous in nature (used for HPN models).  VAR is set when the event is a variable with the common example being a continuous variable where both CONT and VAR are set.

  \begin{verbatim}
    #define UNKNOWN      0
    #define IN           1
    #define OUT          2
    #define PASSIVE      4
    #define ACTIVE       8
    #define DUMMY       16
    #define IDLE        32
    #define WORK        64
    #define DONE       128
    #define ISIGNAL    256
    #define RELEVANT   512
    #define SENT      1024
    #define RECEIVED  2048
    #define PUSH      4096
    #define PULL      8192
    #define KEEP     16384
    #define NONINP   32768
    #define ABSTRAC  65536
    #define CONT    131072
		#define VAR     262144
  \end{verbatim}

  \subsubsection{hsl}
  A char * containing the HSL formula for the event.

  \subsubsection{rate}
  This is the flow rate of a continuous transition.  This is used for
  HPNs.

	\subsubsection{lrate/urate}
	The current lower or upper rate of the event.

	\subsubsection{lrange/urange}
	The lower and upper limit on a continuous variable.
	
	\subsubsection{linitrate/uinitrate}
	The lower and upper initial rate for a continuous variable.
	
	\subsubsection{SOP}
	The sum of products form for the enabling condition on the event.
	
	\subsubsection{inequalities}
	The list of inequalities assigned to the event.

  \subsubsection{immediate, process, data}
  Not used by most developers.

  \subsection{signalADT}
  \begin{verbatim}
    typedef struct signal_struct {
      char * name;
      bool is_level;
      int event;
      int riselower;
      int riseupper;
      int falllower;
      int fallupper;
      int maxoccurrence;
   } *signalADT;
  \end{verbatim}

  \subsubsection{name}
	The name of the signal.
	
	\subsubsection{event}
	

  \subsection{*level\_exp}
  \begin{verbatim}
    typedef struct bool_exp {
      char * product;
      struct bool_exp *next;
    } *level_exp;
  \end{verbatim}

  The level\_exp struct is used to create SOP and POS style boolean
  expressions.  This is done using sequences of 1,0, and -.  1
  indicates true; 0 indicates false; - indicates a don't care.  For
  instance, a circuit with signals abcd and being represented in
  the SOP form where the expression is ac' would be represented as
  1-0-.  
  
  \subsubsection{product}
  Product is a char* representation of the boolean term.  In the SOP
  representation the product variable represents the product term and
  in the POS form the product variable represents the sum term.
  
  \subsubsection{next}
  Next is simply a pointer to the next item in the list.

	\subsection{ineqADT}
	\begin{verbatim}
		typedef struct ineq_tag {
		  intptr_t place;
		  int type;
		  int uconstant;
		  int constant;
		  int signal;
		  intptr_t transition;
		  struct ineq_tag *next;
		} *ineqADT;
	\end{verbatim}
	
	\subsubsection{place}
	An integer that indexes into the EventADT and corresponds to the place being used on the LHS of the inequality.
	
	\subsubsection{type}
	An integer representing the type of inequality.  The list is shown below.
	
	\begin{verbatim}
		0 = >
		1 = >=
		2 = <
		3 = <=
		4 = =
		5 = :=
		6 = 'dot:=
		7 = :=T/F
	\end{verbatim}
	
	\subsubsection{uconstant}
	The upper bound for the inequality when appropriate (for instance in an assignment).
	
	\subsubsection{constant}
	The lower bound or constant value for the LHS of the inequality.
	
	\subsubsection{signal}
	An index into the signalADT for appropriate predicates.
	
	\subsubsection{transition}
	An index into EventADT indicating to which transition the inequality corresponds.
	
	\subsubsection{next}
	A pointer to the next inequality in the list.

  \subsection{clockkeyADT}
  \begin{verbatim}
    typedef struct clockkey {
      int enabled;
      int enabling;
      int eventk_num;
      int causal;
      anti_causal_list anti;
			int lrate;
			int urate;
    } *clockkeyADT;
  \end{verbatim}
  
  \subsubsection{enabling}
  Enabling is an integer index into the rules matrix.  This is the
  index into the row and represents the event or place that has
  enabled this rule.
  
  \subsubsection{enabled}
  Enabled is an integer index into the rules matrix.  This is the
  index into the column and represents the event or place that this
  rule enables.

	\subsubsection{lrate/urate}
	When the data structure contains a continuous variable these integers contain the lower and upper rate for the variable.  These rates are checked by add\_state as they are necessary for differentiating between states.

  \subsubsection{eventk\_num, causal, anti}
  Not used by most developers.

  \section{Transformations}
  %\documentstyle[11pt,epsfig,psfig,times]{article} 

%\textheight 9in
%\topmargin 1in
%\footheight 0.5in
%\textwidth 6.8in
%\oddsidemargin -0.2in

%\usepackage{latex8}
%\usepackage{psfig}
%\usepackage{epsfig}
%\usepackage{latexsym}
%\usepackage{amssymb}

%\begin{document}

Let ${\tt preset}(t)$ and ${\tt postset}(t)$ denote the preset and postset of
a transition $t$.  ${\tt preset}(p)$ and ${\tt postset}(p)$ are similarly 
defined for a place $p$.   


\subsection{Transform 0: Merge Parallel Places}
\label{reduce0}

\begin{figure}[tbh]
\begin{center}
\begin{tabular}{cc}
\psfig{figure=xform0-a,width=27.5mm} \hspace{5mm} &
\psfig{figure=xform0-b,width=22.5mm} \\
(a) \hspace{6mm} & (b)
\end{tabular}
{\caption{\label{xform0}Transform 0. Merge parallel places.}}
\end{center}
\end{figure}

Figure~\ref{xform0}(a) shows two places in parallel.  They can be
merged into a single place as shown in Figure~\ref{xform0}(b) if the following requirements hold:
\begin{itemize}
\item {\tt preset(p1) = preset(p2)}.  
\item {\tt postset(p1) = postset(p2)}.  
\item {\tt marking(p1) = marking(p2)}
\end{itemize}

%\clearpage

\subsection{Transform 1: Remove a Place in a Self Loop}
\label{reduce}

\begin{figure}[!tbh]
\begin{center}
\begin{tabular}{cc}
\psfig{figure=xform1-a,width=31.25mm} \hspace{5mm} &
\psfig{figure=xform1-b,width=35mm} \\
(a) \hspace{6mm} & (b)
\end{tabular}
{\caption{\label{xform1}Transform 1. Remove a self loop on a
    transition.}}
\end{center}
\end{figure}

Figure~\ref{xform1}(a) shows a transition $t$ with a self loop, where a 
place $p$ is in ${\tt preset}(t)$ and ${\tt postset}(t)$, and is marked.
The transformation removes $p$ and its incoming and outgoing flow relations.
Then, the upper bound delay of the flow relations entering $t$ are 
modified such that the upper bound is the maximum of the current max
and that of the flow relation between $p$ and $t$.
The new TPN is shown in Figure~\ref{xform1}(b).  If there is only the
same one place in the preset and postset of $t$, this place is not
removed as shown in Figure~\ref{bad}(c). 

%\clearpage 

\subsection{Transform 2: (BAD TRANSFORM) Remove a Transition in a Loop}
\label{reduce1}

\begin{figure}[tbh]
\begin{center}
\begin{tabular}{cc}
\psfig{figure=xform2-a,width=27.5mm} \hspace{5mm} &
\psfig{figure=xform2-b,width=22.5mm} \\
(a) \hspace{6mm} & (b)
\end{tabular}
{\caption{\label{xform2}Transform 2. Remove a transition in a loop.}}
\end{center}
\end{figure}

Figure~\ref{xform2}(a) shows a loop formed by a transition $t_3$, where the size
of the preset and postset of $t_3$ is 1. The semantics
of this TPN can be interpreted as transition $t_4$ can fire after the
infinite number of times of firings of transition $t_2$ and $t_3$.
If ${\tt u}(p_1) + {\tt u}(p_2) > 0$, by changing ${\tt u}(p2)$ to $\infty$, 
the timing of $t_4$ is preserved.  The new TPN after removing transition 
$t_3$ and its incoming and outgoing flow relations is shown in 
Figure~\ref{xform2}(b).

The code implemented the above transformation is in the function call 
{\em xform\_1} in postproc.cc.  

%\clearpage

\subsection{Transform 3: Remove a Transition With a Single Place in the Postset}
\label{reduce2}

\begin{figure}[tbh]
\begin{center}
\begin{tabular}{cc}
\psfig{figure=xform3a-a,width=35mm} \hspace{10mm} &
\psfig{figure=xform3a-b,width=50mm} \\
(a) \hspace{10mm} & (b)
\end{tabular}
{\caption{\label{xform3a}Transform 3a. Remove the transition with
    the size of postset of 1.}}
\end{center}
\end{figure}

\begin{figure}[tbh]
\begin{center}
\begin{tabular}{cc}
\psfig{figure=xform3b-a,width=37.5mm} \hspace{10mm} &
\psfig{figure=xform3b-b,width=62.5mm} \\
(a) \hspace{10mm} & (b)
\end{tabular}
{\caption{\label{xform3b}Transform 3b. The transformation in
    Figure~\ref{xform3a} without the restriction that places in preset(t)
    have postset sizes of 1.}}
\end{center}
\end{figure}

\begin{figure}[tbh]
\begin{center}
\begin{tabular}{cc}
\psfig{figure=xform3c-a,width=35mm} \hspace{10mm} &
\psfig{figure=xform3c-b,width=57.5mm} \\
(a) \hspace{10mm} & (b)
\end{tabular}
{\caption{\label{xform3c}Transform 3c.  The transformation in
    Figure~\ref{xform3a} without the restriction that the place in
    postset(t) have preset size of 1.}}
\end{center}
\end{figure}

Figure~\ref{xform3a}(a) shows a transition $t$ with only one place in its
postset.  Suppose $t$ is removed.  The incoming and outgoing flow relations
of $t$ are also removed.  Then the place in ${\tt postset}(t)$ is 
merged with those in ${\tt preset}(t)$.  The delay bounds of the place in 
${\tt postset}(t)$ are added to the delay bounds of the places in 
${\tt preset}(t)$, as shown in Figure~\ref{xform3a}(b).
The transformation in Figure~\ref{xform3a}  has the restrictions that
for each place $p$ in
${\tt preset}(t)$, the size of ${\tt postset}(p)$ equals 1 and  for the
place $p$ in ${\tt postset}(t)$, the size of ${\tt preset}(p)$ equals
1.  Figures \ref{xform3b} and \ref{xform3c} show the same
transformations without each restriction respectively.

Figure~\ref{bad}(a) and (b) show where these above transformations
cannot be applied.

All places in preset(t) must have same marking (all marked or all
unmarked) and place in postset(t) can be marked.  If either preset(t)
or postset(t) is marked, new combined places are marked.

%\clearpage

\subsection{Transform 4: Remove A Transition With A Single Place in the Preset}
\label{xform-3}

\begin{figure}[tbh]
\begin{center}
\begin{tabular}{cc}
\psfig{figure=xform4a-a,width=40mm} \hspace{10mm} &
\psfig{figure=xform4a-b,width=41.25mm} \\
(a) \hspace{10mm} & (b)
\end{tabular}
{\caption{\label{xform4a}Transform 4a. Remove a transition with
    preset of size 1.}}
\end{center}
\end{figure}

\begin{figure}[tbh]
\begin{center}
\begin{tabular}{cc}
\psfig{figure=xform4b-a,width=37.5mm} \hspace{10mm} &
\psfig{figure=xform4b-b,width=40mm} \\
(a) \hspace{10mm} & (b)
\end{tabular}
{\caption{\label{xform4b}Transform 4b. Transformation in Figure~\ref{xform4a}
    without restriction that the places in preset(t) have a preset of
    size 0.}}
\end{center}
\end{figure}

\begin{figure}[tbh]
\begin{center}
\begin{tabular}{cc}
\psfig{figure=xform4c-a,width=40mm} \hspace{10mm} &
\psfig{figure=xform4c-b,width=45mm} \\
(a) \hspace{10mm} & (b)
\end{tabular}
{\caption{\label{xform4c}Transform 4c. Transformation in Figure~\ref{xform4a}
    without restriction that the places in postset(t) each have a preset of
    size 1.}}
\end{center}
\end{figure}

Figure~\ref{xform4a}(a) shows a transition $t$ with only one place in its
preset.  Suppose $t$ is removed.  The incoming and outgoing flow relations
of $t$ are also removed.  The the place in ${\tt preset}(t)$ is 
merged with those in ${\tt postset}(t)$.  The delay bounds of the place in 
${\tt preset}(t)$ are added to the delay bounds of the places in 
${\tt postset}(t)$, as shown in Figure~\ref{xform4a}(b).
The transformation in Figure~\ref{xform4a} as the restrictions that
for the place $p$ in
${\tt preset}(t)$, the size of ${\tt postset}(p)$ equals 1 and for each
place $p$ in ${\tt postset}(t)$, the size of ${\tt preset}(p)$ equals
1, too.  Figures~\ref{xform4b} and \ref{xform4c} demonstrate the same
transform where these restrictions are loosened.

The above transformations are also 
restricted by the conditions shown in Figure~\ref{bad}(a) and (b).

All places in postset(t) must have same marking (all marked or all
unmarked) and place in preset(t) can be marked.  If anything is
marked, new combined places are marked.

%\clearpage

\subsection{Transform 5: Merge Transitions With the same preset and postset}
\label{merge-1}

\begin{figure}[tbh]
\begin{center}
\begin{tabular}{cc}
\psfig{figure=xform5a-a,width=58.75mm} \hspace{10mm} &
\psfig{figure=xform5a-b,width=48.75mm} \\
(a) \hspace{10mm} & (b)
\end{tabular}
{\caption{\label{xform5a}Transform 5a. Merge transitions with the same
    preset and postset}}
\end{center}
\end{figure}

\begin{figure}[tbh]
\begin{center}
\begin{tabular}{cc}
\psfig{figure=xform5b-a,width=48.75mm} \hspace{10mm} &
\psfig{figure=xform5b-b,width=46.25mm} \\
(a) \hspace{10mm} & (b)
\end{tabular}
{\caption{\label{xform5b}Transform 5b. Merge transitions where the
    places in their preset and postset have the same preset and
    postset.}}
\end{center}
\end{figure}

Figure~\ref{xform5a}(a) shows two transitions $t_6$ and $t_7$ with 
the same preset and postset.  Those two transitions can be merged together
without affecting the behavior on transitions $t_{10}$, $t_{11}$, and 
$t_{12}$ as shown in Figure~\ref{xform5a}(b).  Since no places are
being removed, no marking restrictions apply.

%Note: Transform 5b is not turned on by default.  It was found to have
%minimal effect except for in a few examples and was computationaly
%intensive.  Additionally, the correctness of this transformation in
%its current form is in question.

Another case is where the two transitions do not have the same preset
and postset, but the places in their preset and postset have the same
preset and postset as shown in Figure~\ref{xform5b}(a).  The two
transitions can still be merged along with the places in their preset
and postset. This transformation is depicted in Figure~\ref{xform5b}.
In order for the transformation to be applied certain timing
constraints must be satisfied.  Specifically, arcs between the places
in the postset to transitions in the postsets of the postset must have
the same timing bounds.  Furthermore, the place to transition arcs
feeding the transitions to be merged must have the same timing
constraints.

%There is also requirement on this transformation that the places in
%the preset of the merged two transitions must have same postset
%excluding the merging transitions, and the places in the postset of
%the merged two transitions must have same preset excluding the merging
%transitions.  For example, in Figure~\ref{xform5b}(a), the places in
%the preset of $t_3$ and $t_4$ are $p_1$ and $p_2$.  The postset of
%$p_1$ and $p_2$ contains only $t_2$, excluding $t_3$ and $t_4$.  The
%places in the postset of $t_3$ and $t_4$ are $p_3$ and $p_4$.  The
%preset of $p_3$ and $p_4$ contains only $t_5$, excluding $t_3$ and
%$t_4$.

%\clearpage

\subsection{Transform 6: Merge Transitions With the same preset}
\label{merge-2}

\begin{figure}[tbh]
\begin{center}
\begin{tabular}{cc}
\psfig{figure=xform6-a,width=57.5mm} \hspace{5mm} &
\psfig{figure=xform6-b,width=50mm} \\
(a) \hspace{5mm} & (b)
\end{tabular}
{\caption{\label{xform6}Transform 6. Merge transitions with the same
    preset.}}
\end{center}
\end{figure}

Figure~\ref{xform6}(a) shows two transitions $t_1$ and $t_2$ with 
the same preset.  It is also required that the places in the postset of 
$t_1$ and $t_2$ must have the same preset aside from the two
transitions being merged.  Additionally, there can only be one place
in the postsets of t1 and t2.  Those two 
transitions can be merged together without affecting the behavior on 
transitions $t_3$, $t_4$, and $t_5$ as shown in Figure~\ref{xform6}(b).

Since the places in the postset of the merging transitions will be
merged, they must have a consistent marking.  The newly merged place
will have the same marking as the places being merged.
%\clearpage

\subsection{Transform 7: Merge Transitions With the same postset}
\label{merge-3}

\begin{figure}[tbh]
\begin{center}
\begin{tabular}{cc}
\psfig{figure=xform7-a,width=51.25mm} \hspace{10mm} &
\psfig{figure=xform7-b,width=51.25mm} \\
(a) \hspace{10mm} & (b)
\end{tabular}
{\caption{\label{xform7}Transform 7. Merge transitions with the same
    postset.}}
\end{center}
\end{figure}

Figure~\ref{xform7}(a) shows two transitions $t_1$ and $t_2$ with 
the same postset.  It is also required that the places in the preset of 
$t_1$ and $t_2$ must have the same postset aside from $t_1$ and $t_2$.
Those two 
transitions can be merged together without affecting the behavior on 
transitions $t_3$, $t_4$, and $t_5$ as shown in
Figure~\ref{xform7}(b).  Additionally, the transitions being merged
may each only have one place in their presets.  The places in the
preset of the transitions
being merged should have consistent markings.  The newly merged place
in the transformed net should have the same marking as the places that
are merged.



%\clearpage

\subsection{Illegal Transformations}

\begin{figure}[tbh]
\begin{center}
\begin{tabular}{cc}
\psfig{figure=bad-1,width=25mm} \hspace{5mm} &
\psfig{figure=bad-2,width=32.5mm} \\
(a) \hspace{6mm} & (b) \\
\psfig{figure=bad-3,width=20mm} \hspace{5mm} &
\psfig{figure=bad-4,width=12.5mm} \\
(c) \hspace{6mm} & (d)
\end{tabular}
{\caption{\label{bad}Illegal transformations.}}
\end{center}
\end{figure}

Figure~\ref{bad} shows several illegal transformations. In Figure~\ref{bad}(a),
neither $t_1$ nor $t_2$ can be removed using transformations in 
section~\ref{reduce1} and \ref{reduce2}.  
They may be merged by transformations insection~\ref{merge-1}, 
\ref{merge-2}, or \ref{merge-3}.  In Figure~\ref{bad}(b), $t_3$ cannot be
removed in that it is impossible to have two places connected without a
transition in between. Figure~\ref{bad}(c) shows a stand-alone self loop which
is not removed for some reason.  I think it can be removed if the wire of 
that transition is not used in any levels.  Figure~\ref{bad}(d) shows a
dangling transition $t_2$.  It cannot be removed because safety failure
maybe hidden; otherwise.

%\clearpage

\subsection{To Do and Notes}

\begin{itemize}
\item Transforms 3d and 4d.  Same As 3a and 4a but with both
  restrictions removed.
\end{itemize}

%\end{document}
\end{document}
