%%%%%%%%%%%%%%%%%%%%%%%%%%%%%%%%%%%%%%%%%%%%%%%%%%%%%%%%%%%%%%%%%%%%%%%%%%%%%%%
%% Copyright (c) 2002 Scott Little <little@eng.utah.edu>
%%
%% File Name     : user.tex
%% Author        : Scott R. Little
%% E-mail        : little@eng.utah.edu
%% Date Created  : 11/01/2002
%%
%% Description   : An ATACS user manual.
%%
%% ToDo          :
%%
%%%%%%%%%%%%%%%%%%%%%%%%%%%%%%%%%%%%%%%%%%%%%%%%%%%%%%%%%%%%%%%%%%%%%%%%%%%%%%%

\documentclass[titlepage,11pt]{article}
\textwidth 6.5in
\textheight 9in
\oddsidemargin -0.2in
\topmargin -0.5in
\usepackage{indentfirst,graphics,alltt,epsfig,color}

\title{The ATACS User's Manual}
\date{Created: November 1, 2002\\
  Last Revised: December 2, 2002
}

\begin{document}
  \maketitle

  %show only subsection granularity in the toc
  \setcounter{tocdepth}{2} 
  
  \tableofcontents
  \newpage
  
  \setlength{\parindent}{0em}
  \setlength{\parskip}{10pt}

  \section{Introduction \& Purpose}  
  This document intends to cover the material required to use ATACS.
  Basic understanding of asynchronous circuits and CAD tools is
  assumed.  The main focus of this document will be the command line
  operation of ATACS.  A web based and graphical version of ATACS do
  exist, but if the command line version is well understood then using
  the other versions is not difficult.
  
  \section{Basic Operation}
  ATACS is a CAD tool designed for synthesis and/or verification of
  timed circuits.  The command line version of ATACS has two modes.
  There is an interactive and noninteractive mode.  To enter
  interactive mode simply execute the ATACS binary from the command
  line.  A prompt will appear and then commands and options can be
  given and executed one at a time.  To get a list of commands type
  help and hit enter.   A list of valid commands are then listed.
  Typing quit and hitting enter will exit the program.  The available
  commands and options are explained in detail in the next section.

  To use the noninteractive mode the ATACS binary is run with a number
  of options and commands followed by any number of input files.  If
  you are used to the standard Unix command structure ATACS will
  operate a bit differently than you might first expect.  The reason
  for this difference is that the command line flags are not strictly
  options.  Both commands and options can be give to ATACS on the
  command line.

  \subsection{Environment Variables}
  There are a few environment variables that are required for certain
  ATACS functionality.  ATACS does use some third party applications
  which require other environment variables.  Discussion of these
  programs can be found in the Developer's Manual if more information
  is required.

  The two environment variables that should be set to run ATACS are
  the ATACS and ATACSHELP variables.  These should be set to your top
  level ATACS directory.  For example, here are some lines that can be
  modified if necessary and added to your .cshrc file.  If you use
  another shell you should be able to figure out how to adapt the
  lines.
  \begin{verbatim}
    setenv ATACS ${HOME}/atacs
    setenv ATACSHELP ${ATACS}
  \end{verbatim}
  
  \section{Commands vs. Options}
  Using ATACS in noninteractive mode is the most common way to use
  ATACS among the developers because it lends itself to easily
  scripting common operations.  In some respects the ATACS command
  line is a scripting language in itself.  The command line flags are
  a combination of commands and options.  When a command is parsed it
  is executed with the current options.  When the command returns
  ATACS will continue processing the command line and executing the
  remaining command line flags.  Each of the commands and options will
  be described followed by a short section of examples to help clarify
  the material.
  
  \subsection{Commands}
  Commands are simply actions that ATACS will perform when that
  particular command is parsed.
  
  \subsubsection{load, -l?}
  The load command is used to load an external specification into the
  internal ATACS data structures in preparation for processing.
  Currently ATACS supports the following input formats:
  CSP(-lc,loadcsp), HSE(-lh,loadhse), VHDL(-lv,loadvhdl),Petri
  Nets(-lg,loadg), Level-Ruled Petri Nets(-ln,loadlpn), Extended Burst
  Mode Machines(-lx,loadxbm), ER(-le,loader), Timed Event Level
  structures(-lt,loadtel), and RSG(-ls,loadrsg).  These formats may
  not be completely clear, but they are described later in this
  document in the file formats section.
  
  \subsubsection{store, -s?}
  The store command is used to tell ATACS how to store the results it
  generates.  The formats here do overlap a bit with input formats
  which can be useful.  The supported output formats are: Petri
  Net(-sg,storeg), ER(-se,storeer), RSG(-ss,storersg), State
  Graph(-sm,storesg), PRS(-sp,storeprs), Net(-sn,storenet), and
  VHDL(-sv,storevhdl).  These formats are described more completely
  later in this document in the file formats section.
  
  \subsubsection{display, -d?}
  The display command is used to display the different types of graph
  structures using Tom Rokicki's parg program or Bell Lab's Dot
  program.  Setup and installation of these programs is described in
  the Developer's Manual.  All of the options can be displayed with
  both programs except for NET which is only viewable using dot.  The
  supported display options are: ER(-de,shower), LPN(-dl,showlpn),
  RSG(-ds,showrsg), STATS(-dz,showstat), Net(-dn,shownet).  STATS is
  an RSG that has been annotated with the steady state probability of
  being in each state.  It is also important to note that Net in the
  display section is different from Net format used in store.  The Net
  format used in the display section actually displays a diagram of
  the current circuit.
  
  \subsubsection{print, -p?}
  The print command is used to display and print the different types
  of graph structures using Tom Rokicki's parg program or Bell Lab's Dot
  program.  The supported print commands are: ER(-pe,printer),
  LPN(-pl,printlpn), RSG(-ps,printrsg), (-pz,printstat),
  REG(-pr,printreg), NET(-pn,printnet).  The STAT and NET options are
  the same types as are used in the display section.  The REG option
  is new and is used to print grf files for each signal and each type
  of region associated with the signal.  REG can only be displayed
  using Parg.
  
  \subsubsection{synthesis, -y?}
  The synthesis command allows the user to select the synthesis method
  to be used.  The valid synthesis commands are: single(-ys,single),
  multicube(-ym,multicube), bdd(-yb,bdd), direct(-yd,direct).  The
  single method uses a single-cube algorithm for synthesis which
  attempts to find a single cube implementation for each excitation
  region.  The multicube method uses a multi-cube algorithm where
  primes are found using espresso and candidate primes are added when
  necessary to find an optimal cover.  The bdd method uses BDDs to
  represent all possible covers.  The direct method for synthesis
  doesn't generate a state graph, but uses the heuristic timing
  analysis.  The heuristic command should also be executed with this
  synthesis command.
  
  \subsubsection{analysis, -a?}
  The analysis commands are used to compute statistical information
  about the cycle period of the circuit.  The valid commands are:
  simcyc(-as,simcyc), and profile(-ap,profile).  The simcyc command
  runs a user specified number of random simulations to determine the
  average cycle period of the design.  The results are written to a
  \verb1<1design\verb1>1.stats file.  This file contains the
  observations computed from each simulation run and a set of summary
  statistics.  The profile command simply computes the cycle period
  profile.  
  
  \subsubsection{verify, -v?}
  The verify command will verify certain properties about the circuit
  or specification.  The valid verify commands are: atacs(-va,verify),
  orbits(-vo,verorbits), and search(-vs,search).  
  
  \subsection{Options}

  \subsubsection{timing methods, -t?}
  There are six possible timing methods to choose from:
  untimed(-tu,untimed), geometric(-tg,geometric), POSETs(-ts,posets),
  BAG(-tb,bag), BAP(-tp,bap), and BAPTDC(-tt,baptdc).  The untimed
  method ignores all timing information which has been provided and
  synthesizes a speed-independent circuit.  The geometric method uses
  geometric regions (or zones) to represent the timing information
  during state space exploration. The POSETs method also uses zones,
  but it also uses a POSET matrix to track timing relationships
  between events. The usual result is that larger zones are produced
  resulting in a smaller number of zones being needed to represent the
  timed state space.  This typically results in much faster and memory
  efficient timed state space exploration.  The BAG (Bourne Again
  Geometric) method is the same as the geometric method except that it
  uses a time-independent choice semantics.  In other words, when
  there is a choice in the specification, the choice is made first and
  only then is a timing assignment made.  The BAP (Bourne Again
  POSETs) method is similar to POSETs except that it uses only one
  matrix to represent each time state, the POSET matrix, rather than
  two.  The BAP method is time-independent while BAPTDC (BAP
  Time-Dependent Choice) uses a time-dependent choice semantics.

  \subsubsection{timing options, o?}
  There are three possible timing options: abstract(-oa,abstract),
  po(-op,pored), and nofail(-of,nofail).  The abstract method allows
  for irrelevant environment information to be marked and abstracted
  away which allows for more efficient timing analysis.  The po method
  turns on the partial order reduction method to improve the
  efficiency of timing analysis for synthesis and verification.
  
  \subsubsection{technology options, -o?}
  There are five possible target technologies: atomic
  gates(-ot,atomic), generalized-C(-og,gc), standard-C(-os,gatelevel),
  burst-mode gC(-ob,burstgc), and burst-mode 2-level(-obol,burst2l).  An
  atomic gate is one in which all delay can be assumed to be at the
  output of the logic function.  Using this technology, one
  combinational logic gate is created for each output signal.  A
  generalized C-element is a gate which is composed of an arbitrary
  sum-of-products function to pullup the signal as well as an arbitrary
  pulldown network.  The generalized-C switch states that you are
  mapping to such gates.  These gC gates can be decomposed into ANDs,
  ORs, and a single Muller C-element.  This structure is called a
  standard C-element and requires differenct correctness conditions to
  guarantee hazard-freedom on all internal signals. Selecting standard-C
  produces correct timed circuits for libraries with only ANDs, ORs, and
  a C-element.  The burst-mode gC options produces correct generalized
  C-element circuits under burst-mode assumptions while burst-mode
  2-level produces correct 2-level sum of products solutions also under
  burst-mode assumptions.
  
  \subsubsection{other options, -o?}
  There are two other options: dot(-od,dot) and verbose(-ov,verbose).
  The dot option tells ATACS to use the Bell Labs DOT tool for
  displaying graphs.  The verbose option causes ATACS to save all
  intermediate files not only just the final result.
  
  \subsubsection{miscellaneous}
  There are two remaining options which affect circuit parameters:
  maximum fan-in for a gate(-M\#,maxsize) and default gate
  delay(-G\#/\#,gatedelay).  The maximum fan-in is a measure of the
  allowable number of transistors in a single stack in a gC.  The
  default value is 4.  The default gate delay sets the lower and upper
  timing bounds on the delay of a gate.  The default values are <0,1>.
  
  \subsection{Examples}
  
  \section{File Types}
  ATACS uses a number of different file types. There are csp files,
  handshaking expansion files (HSE), VHDL files, signal transition
  graph files (G), burst-mode state machine files (UNC), timed
  event-rule structure files (ER), timed event/level structure file
  (TEL), and a reduced state graph file (RSG).  Each type is described
  in detail below.
  
  \subsection{CSP Files}
  A brief description of the constructs found in CSP files is given below.
 
  \subsubsection{Modules}
  A module is composed of a name, set of delay macros, signal
  declarations, and process statements.  For example:
  
  \begin{quote}
    
    /* A simple inverter */\\
    \textbf{module} mycircuit;
    
    \textbf{delay} mygates = \verb!<!2,3; 1,2\verb!>!;
    
    \textbf{input} a;\\
    \textbf{output} b = {true, mygates};\\
    
    \textbf{process} inverter;\\
    \verb1*1[ [a+]; b-; [a-]; b+ ]\\
    endprocess\\
      
    \textbf{process} environment;\\
    \verb1*1[ a+; [b-]; a-; [b+] ]\\
    \textbf{endprocess}\\
    \textbf{endmodule}
    
  \end{quote}
  
  NOTE: text enclosed in /* and */  are comments.
  
  \subsubsection{Delay macros}
  A delay macro is composed of the keyword delay, an identifier, a lower
  and upper bound for rising transitions, and a lower and upper bound
  for falling transitions.  For example:
  
  \begin{quote}
    \textbf{delay} mygates = \verb1<12,3; 1,2\verb1>1;
  \end{quote}
  
  This delay macro is used to specify signals which have rising signal
  transtions that occur between 2 and 3 time units after becoming
  enabled, and falling transitions that occur between 1 and 2 time units
  after becoming enabled.
  
  \subsubsection{Signal declarations}
  A signal declaration is composed of a type (either it is an
  \textbf{input} or an \textbf{output} signal), an identifier, an intial
  value for the signal (either \textbf{true} or \textbf{false}, where
  \textbf{false} is the default), and a delay specification (where
  \verb1<10,infinity\verb1>1 is the default).  For example:
  
  \begin{quote}
    \textbf{input} a = {\textbf{false}, mygates}; /* a is an input
    signal which is initially false with a delay of 0 to infinity. */
    
    \textbf{output} b = {\textbf{true}, mygates}; /* b is an output
    signal which is initially true with a delay specified by the delay
    macro mygates. */
  \end{quote}
  
  \subsubsection{Process statements}
  A process statement is composed of a name and a sequence of statements
  describing the behavior of the process. These statements can be
  combined either in sequence ";" or in parallel "\verb1||1".  There are two
  basic types of statements: signal transitions and guarded commands.
  For each signal s, there are two signal transitions s+ which indicates
  that s goes from false to true and s- which indicates that s goes from
  true to false.  A guarded command is of the form:
  
  \begin{quote}
    [ guard1 -\verb1>1 statements1\\
      \verb1|1 guard2 -\verb1>1 statements2\\
      \verb1|1 etc.\\
    ]
  \end{quote}
  
  Each guard is a boolean condition which when true allows the
  statements following it to occur.  For example, the following
  guarded command states that if signals a and b both rise, then c
  is set to true.  Otherwise, if either c or d fall, then e is set
  to true.
  
  \begin{quote}
    [ a+ \& b+ -\verb1>1 c+\\
      \verb1|1 c- \verb1|1 d- -\verb1>1 e+ ]
  \end{quote}
  
  Guarded commands can also loop back to the beginning of the statement
  when the list of statements ends with a "*".  For example,
  
  \begin{quote}
    [ done- -\verb1>1 req+; [ack+]; req-; [ack-]; *\\
      \verb1|1  done+ -\verb1>1 skip\\
    ]; end+
  \end{quote}
  
  The guarded commands above loops until done rises, then it sets end to
  true.  Finally, there are a couple of shorthands of interest.  The
  first is [guard -\verb1>1 skip] is abbreviated as [guard]. The second is
  [true -\verb1>1 statements;*] is abbreviated as *[statements].  For example,
  
  \begin{quote}
    \textbf{process} inverter;
    
     [ true -\verb1>1 [a+ -\verb1>1 skip]; b-; [a- -\verb1>1 skip]; b+;
       \verb1*1 ]
     
     \textbf{endprocess}
  \end{quote}
  
  Can be rewritten as:
  
  \begin{quote}
    \textbf{process} inverter;\\
    \verb1*1[ [a+]; b-; [a-]; b+ ]\\
    \textbf{endprocess}
  \end{quote}
  
  \subsection{HSE Files}
  A brief description of the constructs found in HSE files is given below.
  
  \subsubsection{Modules}
  A module is composed of a name, set of delay macros, signal
  declarations, and process statements.  For example:
  
  \begin{quote}
    /* A simple inverter */\\
    \textbf{module} mycircuit;
    
    \textbf{delay} mygates = \verb1<12,3; 1,2\verb1>1;
    
    \textbf{input} a;\\
    \textbf{output} b = {\textbf{true}, mygates};
    
    \textbf{process} inverter;\\
    \verb1*1[ [a]; b-; [\verb1~1a]; b+ ]\\
    \textbf{endprocess}
    
    \textbf{process} environment;\\
    \verb1*1[ a+; [\verb1~1b]; a-; [b] ]\\
    \textbf{endprocess}\\
    \textbf{endmodule}
  \end{quote}
  
  NOTE: text enclosed in /* and */  are comments.
  
  \subsubsection{Delay macros}
  A delay macro is composed of the keyword delay, an identifier, a lower
  and upper bound for rising transitions, and a lower and upper bound
  for falling transitions.  For example:
  
  \begin{quote}
    \textbf{delay} mygates = \verb1<12,3; 1,2\verb1>1;
  \end{quote}
  
  This delay macro is used to specify signals which have rising signal
  transtions that occur between 2 and 3 time units after becoming
  enabled, and falling transitions that occur between 1 and 2 time units
  after becoming enabled.
  
  \subsubsection{Signal declarations}
  A signal declaration is composed of a type (either it is an
  \textbf{input} or an \textbf{output} signal), an identifier, an intial
  value for the signal (either \textbf{true} or \textbf{false}, where
  \textbf{false} is the default), and a delay specification (where
  \verb1<10,infinity\verb1>1 is the default).  For example:
  
  \begin{quote}
    input a = {false, mygates}; /* a is an input signal which is
    initially false with a delay of 0 to infinity. */

    output b = {true, mygates}; /* b is an output signal which is
    initially true with a delay specified by the delay macro mygates. */
  \end{quote}
  
  \subsubsection{Process statements}
  A process statement is composed of a name and a sequence of statements
  describing the behavior of the process. These statements can be
  combined either in sequence ";" or in parallel "\verb1||1".  There are two
  basic types of statements: signal transitions and guarded commands.
  For each signal s, there are two signal transitions s+ which indicates
  that s goes from false to true and s- which indicates that s goes from
  true to false.  A guarded command is of the form:
  
  \begin{quote}
    [ guard1 -\verb1>1 statements1\\
      \verb1|1 guard2 -\verb1>1 statements2\\
      \verb1|1 etc.\\
    ]
  \end{quote}
  
  Each guard is a boolean condition which when true allows the
  statements following it to occur.  For example, the following guarded
  command states that if signals a and b are both true, then c is set to
  true.  Otherwise, if neither a or b is true, then d is set to true.
  
  \begin{quote}
    [ a \& b -\verb1>1 c+\\
      \verb1|1 \verb1~1a \verb1|1 \verb1~1b -\verb1>1 d+\\
    ]
  \end{quote}
  
  Guarded commands can also loop back to the beginning of the statement
  when the list of statements ends with a "*".  For example,
  
  \begin{quote}
    [ \verb1~1done -\verb1>1 req+; [ack]; req-; [\verb1~1ack]; *\\
      \verb1|1  done -\verb1>1 skip\\
    ]; end+
  \end{quote}
  
  The guarded commands above loops until done is true, then it sets end
  to true.  Finally, there are a couple of shorthands of interest.  The
  first is [guard -\verb1>1 skip] is abbreviated as [guard]. The second is
  [true -\verb1>1 statements;*] is abbreviated as *[statements].  For example,
  
  \begin{quote}
    \textbf{process} inverter;
  
     [ true -\verb1>1 [a -\verb1>1 skip]; b-; [\verb1~1a -\verb1>1 skip]; b+; *
     ]
     
     \textbf{endprocess}
  \end{quote}
  
  Can be rewritten as:
  
  \begin{quote}
    \textbf{process} inverter;\\
    \verb1*1[ [a]; b-; [\verb1~1a]; b+ ]\\
    \textbf{endprocess}
  \end{quote}
  
  \subsection{VHDL Files}
  A brief description of the constructs found in VHDL files is given below.
  
  \subsection{G Files}
  A brief description of the constructs found in G files is given below.

  \subsection{ER Files}

  \subsection{TEL Files}

  \subsection{RSG Files}
  
  \subsection{UNC Files}
  A brief description of the constructs found in UNC files is given below.
  
  \subsection{Action}
  You can select to either do synthesis and verification.  The result of
  synthesis is either a logic description of a circuit as a set of
  production rules that implements the given specification, or an error
  message stating why the circuit could not be produced.  The result of
  verification is either a message that no errors were found during
  state space exploration or a trace that indicates a sequence of
  transitions leading to the error.
  
  \subsection{ATACS errors}
  There are several reasons that ATACS may not be able to produce a
  circuit.  The first is there is a syntax error in your specification.
  If this is the case, a line number where the error is discovered
  should be reported.
  
  The second is the graph generated from the timed handshaking expansion
  may have some problems.  It may not be live which means that there is
  not enough tokens in the graph and the system will deadlock.  It may
  not be safe which indicates that there are too many tokens in the
  graph.  Your specification needs to modified to make it work.  Be sure
  that your initial values for your signals is set properly.
  
  The third is during state space exploration an error may be
  encountered such as an unexpected deadlock.  Again, this indicates a
  problem with your initial specification.
  
  Fourth, your specification may not satisfy complete state coding (or
  CSC).  This means there exists two states with the same binary code
  which have different outputs enabled to change.  In such a
  situation, a circuit cannot be produced as it is unable to tell what
  to do when it is in such a state.  If this is the case, you need to
  either add a state variable to solve the state coding problem,
  rearrange your signal transitions (i.e., reshuffling), or change the
  timing specification to remove the state(s).
  
  Fifth, your specification may be non-persistent.  For example if a+
  causes b+ and c+, but a- can occur after b+ but before c+, it is
  said that a is non-persistent with respect to c+.  In such a
  situation, no single cube cover existes.  To fix this problem, try
  using the BDD synthesis method.

  Finally, you may have a unresolvable conflict.  This indicates again
  that no single-cube cover exists, try using the BDD method.

  \subsection{Production rules}
  After a circuit is synthesized successfully, ATACS outputs a set of
  production rules describing the circuit implementation that is needed
  to realize the specified circuit. The simplest forms are for
  combinational gates.  For example:
  
  [ c: (a \& b)] combinational An AND gate with inputs a and b and output
  c.
  
  [~c: (a \& b)] combinational A NAND gate with inputs a and b and output
  c.
  
  [ c: (a)] combinational The next two represent an OR gate with inputs
  [ c: (b)] combinational a and b and output c.
  
  Often, the implementation uses gates called generalized C-elements (or
  gC).  These gates an aribitrary sum-of-products pullup stack to set
  the signal to true, and an arbitrary sum-of-products pulldown stack to
  set the signal to false.  For example:
  
  [+c:(a \& b)] These two productions describe a Muller C-element.
  
  [-c:(~a \& ~b)] [+c:(a \& b)] These two productions describe a
  gC-element which sets c to true [-c:(~a)] when a and b are true, but
  resets when only a is false.
  
  \section{Flags I use}
  Which flags to test against your stuff w/ and w/out posets?
  
  atacs -oOofoplcsgllva \verb1<1filename\verb1>1\\
  oO - no orbits match (does not make sense with LPNs)\\
  of - ignore failures (important if you want to play with disablings)\\
  op - use partial orders (remove if you don't want partial orders)\\
  lc - load csp file (or lh - load hse, lv - load vhdl)\\
  sg - store Petri net file\\
  ll - load LPN\\
  va - verify\\
  
  Note if file is a .g file then you can skip the lgsg and just do:\\
  atacs -oOofopllva
  
  Which flags to test against Eric's stuff w/ and w/out posets?
  
  atacs -tpofopva \verb1<1filename\verb1>1\\
  tp - use bap timing method\\
  rest are same\\
  
  Which flags to test against my stuff w/ and w/out posets?
  
  Same as mine except begin with -tS, ex.
  
  atacs -tSoOofoplcsgllva \verb1<1filename\verb1>1
  
  -----
  
  I'm debugging my timing method, and usually the problem is with the 
  transformed net rather than the method.  Namely, in my previous email I 
  said that:
  
  atacs -lcsg
  
  or
  
  atacs -lhsg
  
  would translate a csp/hse file to a g file.  
  
  This does not always do a good job of it, and the resulting graph does not 
  have the same behavior.  
  
  To check this, do
  
  atacs -lcde  or atacs -lhde
  
  and compare the resulting net with the one from:
  
  atacs -lldl
  
  If they look different, this may indicate a problem.
  
\end{document}